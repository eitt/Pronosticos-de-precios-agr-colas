
% Default to the notebook output style

    


% Inherit from the specified cell style.




    
\documentclass[11pt]{article}

    
    
    \usepackage[T1]{fontenc}
    % Nicer default font (+ math font) than Computer Modern for most use cases
    \usepackage{mathpazo}

    % Basic figure setup, for now with no caption control since it's done
    % automatically by Pandoc (which extracts ![](path) syntax from Markdown).
    \usepackage{graphicx}
    % We will generate all images so they have a width \maxwidth. This means
    % that they will get their normal width if they fit onto the page, but
    % are scaled down if they would overflow the margins.
    \makeatletter
    \def\maxwidth{\ifdim\Gin@nat@width>\linewidth\linewidth
    \else\Gin@nat@width\fi}
    \makeatother
    \let\Oldincludegraphics\includegraphics
    % Set max figure width to be 80% of text width, for now hardcoded.
    \renewcommand{\includegraphics}[1]{\Oldincludegraphics[width=.8\maxwidth]{#1}}
    % Ensure that by default, figures have no caption (until we provide a
    % proper Figure object with a Caption API and a way to capture that
    % in the conversion process - todo).
    \usepackage{caption}
    \DeclareCaptionLabelFormat{nolabel}{}
    \captionsetup{labelformat=nolabel}

    \usepackage{adjustbox} % Used to constrain images to a maximum size 
    \usepackage{xcolor} % Allow colors to be defined
    \usepackage{enumerate} % Needed for markdown enumerations to work
    \usepackage{geometry} % Used to adjust the document margins
    \usepackage{amsmath} % Equations
    \usepackage{amssymb} % Equations
    \usepackage{textcomp} % defines textquotesingle
    % Hack from http://tex.stackexchange.com/a/47451/13684:
    \AtBeginDocument{%
        \def\PYZsq{\textquotesingle}% Upright quotes in Pygmentized code
    }
    \usepackage{upquote} % Upright quotes for verbatim code
    \usepackage{eurosym} % defines \euro
    \usepackage[mathletters]{ucs} % Extended unicode (utf-8) support
    \usepackage[utf8x]{inputenc} % Allow utf-8 characters in the tex document
    \usepackage{fancyvrb} % verbatim replacement that allows latex
    \usepackage{grffile} % extends the file name processing of package graphics 
                         % to support a larger range 
    % The hyperref package gives us a pdf with properly built
    % internal navigation ('pdf bookmarks' for the table of contents,
    % internal cross-reference links, web links for URLs, etc.)
    \usepackage{hyperref}
    \usepackage{longtable} % longtable support required by pandoc >1.10
    \usepackage{booktabs}  % table support for pandoc > 1.12.2
    \usepackage[inline]{enumitem} % IRkernel/repr support (it uses the enumerate* environment)
    \usepackage[normalem]{ulem} % ulem is needed to support strikethroughs (\sout)
                                % normalem makes italics be italics, not underlines
    

    
    
    % Colors for the hyperref package
    \definecolor{urlcolor}{rgb}{0,.145,.698}
    \definecolor{linkcolor}{rgb}{.71,0.21,0.01}
    \definecolor{citecolor}{rgb}{.12,.54,.11}

    % ANSI colors
    \definecolor{ansi-black}{HTML}{3E424D}
    \definecolor{ansi-black-intense}{HTML}{282C36}
    \definecolor{ansi-red}{HTML}{E75C58}
    \definecolor{ansi-red-intense}{HTML}{B22B31}
    \definecolor{ansi-green}{HTML}{00A250}
    \definecolor{ansi-green-intense}{HTML}{007427}
    \definecolor{ansi-yellow}{HTML}{DDB62B}
    \definecolor{ansi-yellow-intense}{HTML}{B27D12}
    \definecolor{ansi-blue}{HTML}{208FFB}
    \definecolor{ansi-blue-intense}{HTML}{0065CA}
    \definecolor{ansi-magenta}{HTML}{D160C4}
    \definecolor{ansi-magenta-intense}{HTML}{A03196}
    \definecolor{ansi-cyan}{HTML}{60C6C8}
    \definecolor{ansi-cyan-intense}{HTML}{258F8F}
    \definecolor{ansi-white}{HTML}{C5C1B4}
    \definecolor{ansi-white-intense}{HTML}{A1A6B2}

    % commands and environments needed by pandoc snippets
    % extracted from the output of `pandoc -s`
    \providecommand{\tightlist}{%
      \setlength{\itemsep}{0pt}\setlength{\parskip}{0pt}}
    \DefineVerbatimEnvironment{Highlighting}{Verbatim}{commandchars=\\\{\}}
    % Add ',fontsize=\small' for more characters per line
    \newenvironment{Shaded}{}{}
    \newcommand{\KeywordTok}[1]{\textcolor[rgb]{0.00,0.44,0.13}{\textbf{{#1}}}}
    \newcommand{\DataTypeTok}[1]{\textcolor[rgb]{0.56,0.13,0.00}{{#1}}}
    \newcommand{\DecValTok}[1]{\textcolor[rgb]{0.25,0.63,0.44}{{#1}}}
    \newcommand{\BaseNTok}[1]{\textcolor[rgb]{0.25,0.63,0.44}{{#1}}}
    \newcommand{\FloatTok}[1]{\textcolor[rgb]{0.25,0.63,0.44}{{#1}}}
    \newcommand{\CharTok}[1]{\textcolor[rgb]{0.25,0.44,0.63}{{#1}}}
    \newcommand{\StringTok}[1]{\textcolor[rgb]{0.25,0.44,0.63}{{#1}}}
    \newcommand{\CommentTok}[1]{\textcolor[rgb]{0.38,0.63,0.69}{\textit{{#1}}}}
    \newcommand{\OtherTok}[1]{\textcolor[rgb]{0.00,0.44,0.13}{{#1}}}
    \newcommand{\AlertTok}[1]{\textcolor[rgb]{1.00,0.00,0.00}{\textbf{{#1}}}}
    \newcommand{\FunctionTok}[1]{\textcolor[rgb]{0.02,0.16,0.49}{{#1}}}
    \newcommand{\RegionMarkerTok}[1]{{#1}}
    \newcommand{\ErrorTok}[1]{\textcolor[rgb]{1.00,0.00,0.00}{\textbf{{#1}}}}
    \newcommand{\NormalTok}[1]{{#1}}
    
    % Additional commands for more recent versions of Pandoc
    \newcommand{\ConstantTok}[1]{\textcolor[rgb]{0.53,0.00,0.00}{{#1}}}
    \newcommand{\SpecialCharTok}[1]{\textcolor[rgb]{0.25,0.44,0.63}{{#1}}}
    \newcommand{\VerbatimStringTok}[1]{\textcolor[rgb]{0.25,0.44,0.63}{{#1}}}
    \newcommand{\SpecialStringTok}[1]{\textcolor[rgb]{0.73,0.40,0.53}{{#1}}}
    \newcommand{\ImportTok}[1]{{#1}}
    \newcommand{\DocumentationTok}[1]{\textcolor[rgb]{0.73,0.13,0.13}{\textit{{#1}}}}
    \newcommand{\AnnotationTok}[1]{\textcolor[rgb]{0.38,0.63,0.69}{\textbf{\textit{{#1}}}}}
    \newcommand{\CommentVarTok}[1]{\textcolor[rgb]{0.38,0.63,0.69}{\textbf{\textit{{#1}}}}}
    \newcommand{\VariableTok}[1]{\textcolor[rgb]{0.10,0.09,0.49}{{#1}}}
    \newcommand{\ControlFlowTok}[1]{\textcolor[rgb]{0.00,0.44,0.13}{\textbf{{#1}}}}
    \newcommand{\OperatorTok}[1]{\textcolor[rgb]{0.40,0.40,0.40}{{#1}}}
    \newcommand{\BuiltInTok}[1]{{#1}}
    \newcommand{\ExtensionTok}[1]{{#1}}
    \newcommand{\PreprocessorTok}[1]{\textcolor[rgb]{0.74,0.48,0.00}{{#1}}}
    \newcommand{\AttributeTok}[1]{\textcolor[rgb]{0.49,0.56,0.16}{{#1}}}
    \newcommand{\InformationTok}[1]{\textcolor[rgb]{0.38,0.63,0.69}{\textbf{\textit{{#1}}}}}
    \newcommand{\WarningTok}[1]{\textcolor[rgb]{0.38,0.63,0.69}{\textbf{\textit{{#1}}}}}
    
    
    % Define a nice break command that doesn't care if a line doesn't already
    % exist.
    \def\br{\hspace*{\fill} \\* }
    % Math Jax compatability definitions
    \def\gt{>}
    \def\lt{<}
    % Document parameters
    \title{Code}
    
    
    

    % Pygments definitions
    
\makeatletter
\def\PY@reset{\let\PY@it=\relax \let\PY@bf=\relax%
    \let\PY@ul=\relax \let\PY@tc=\relax%
    \let\PY@bc=\relax \let\PY@ff=\relax}
\def\PY@tok#1{\csname PY@tok@#1\endcsname}
\def\PY@toks#1+{\ifx\relax#1\empty\else%
    \PY@tok{#1}\expandafter\PY@toks\fi}
\def\PY@do#1{\PY@bc{\PY@tc{\PY@ul{%
    \PY@it{\PY@bf{\PY@ff{#1}}}}}}}
\def\PY#1#2{\PY@reset\PY@toks#1+\relax+\PY@do{#2}}

\expandafter\def\csname PY@tok@w\endcsname{\def\PY@tc##1{\textcolor[rgb]{0.73,0.73,0.73}{##1}}}
\expandafter\def\csname PY@tok@c\endcsname{\let\PY@it=\textit\def\PY@tc##1{\textcolor[rgb]{0.25,0.50,0.50}{##1}}}
\expandafter\def\csname PY@tok@cp\endcsname{\def\PY@tc##1{\textcolor[rgb]{0.74,0.48,0.00}{##1}}}
\expandafter\def\csname PY@tok@k\endcsname{\let\PY@bf=\textbf\def\PY@tc##1{\textcolor[rgb]{0.00,0.50,0.00}{##1}}}
\expandafter\def\csname PY@tok@kp\endcsname{\def\PY@tc##1{\textcolor[rgb]{0.00,0.50,0.00}{##1}}}
\expandafter\def\csname PY@tok@kt\endcsname{\def\PY@tc##1{\textcolor[rgb]{0.69,0.00,0.25}{##1}}}
\expandafter\def\csname PY@tok@o\endcsname{\def\PY@tc##1{\textcolor[rgb]{0.40,0.40,0.40}{##1}}}
\expandafter\def\csname PY@tok@ow\endcsname{\let\PY@bf=\textbf\def\PY@tc##1{\textcolor[rgb]{0.67,0.13,1.00}{##1}}}
\expandafter\def\csname PY@tok@nb\endcsname{\def\PY@tc##1{\textcolor[rgb]{0.00,0.50,0.00}{##1}}}
\expandafter\def\csname PY@tok@nf\endcsname{\def\PY@tc##1{\textcolor[rgb]{0.00,0.00,1.00}{##1}}}
\expandafter\def\csname PY@tok@nc\endcsname{\let\PY@bf=\textbf\def\PY@tc##1{\textcolor[rgb]{0.00,0.00,1.00}{##1}}}
\expandafter\def\csname PY@tok@nn\endcsname{\let\PY@bf=\textbf\def\PY@tc##1{\textcolor[rgb]{0.00,0.00,1.00}{##1}}}
\expandafter\def\csname PY@tok@ne\endcsname{\let\PY@bf=\textbf\def\PY@tc##1{\textcolor[rgb]{0.82,0.25,0.23}{##1}}}
\expandafter\def\csname PY@tok@nv\endcsname{\def\PY@tc##1{\textcolor[rgb]{0.10,0.09,0.49}{##1}}}
\expandafter\def\csname PY@tok@no\endcsname{\def\PY@tc##1{\textcolor[rgb]{0.53,0.00,0.00}{##1}}}
\expandafter\def\csname PY@tok@nl\endcsname{\def\PY@tc##1{\textcolor[rgb]{0.63,0.63,0.00}{##1}}}
\expandafter\def\csname PY@tok@ni\endcsname{\let\PY@bf=\textbf\def\PY@tc##1{\textcolor[rgb]{0.60,0.60,0.60}{##1}}}
\expandafter\def\csname PY@tok@na\endcsname{\def\PY@tc##1{\textcolor[rgb]{0.49,0.56,0.16}{##1}}}
\expandafter\def\csname PY@tok@nt\endcsname{\let\PY@bf=\textbf\def\PY@tc##1{\textcolor[rgb]{0.00,0.50,0.00}{##1}}}
\expandafter\def\csname PY@tok@nd\endcsname{\def\PY@tc##1{\textcolor[rgb]{0.67,0.13,1.00}{##1}}}
\expandafter\def\csname PY@tok@s\endcsname{\def\PY@tc##1{\textcolor[rgb]{0.73,0.13,0.13}{##1}}}
\expandafter\def\csname PY@tok@sd\endcsname{\let\PY@it=\textit\def\PY@tc##1{\textcolor[rgb]{0.73,0.13,0.13}{##1}}}
\expandafter\def\csname PY@tok@si\endcsname{\let\PY@bf=\textbf\def\PY@tc##1{\textcolor[rgb]{0.73,0.40,0.53}{##1}}}
\expandafter\def\csname PY@tok@se\endcsname{\let\PY@bf=\textbf\def\PY@tc##1{\textcolor[rgb]{0.73,0.40,0.13}{##1}}}
\expandafter\def\csname PY@tok@sr\endcsname{\def\PY@tc##1{\textcolor[rgb]{0.73,0.40,0.53}{##1}}}
\expandafter\def\csname PY@tok@ss\endcsname{\def\PY@tc##1{\textcolor[rgb]{0.10,0.09,0.49}{##1}}}
\expandafter\def\csname PY@tok@sx\endcsname{\def\PY@tc##1{\textcolor[rgb]{0.00,0.50,0.00}{##1}}}
\expandafter\def\csname PY@tok@m\endcsname{\def\PY@tc##1{\textcolor[rgb]{0.40,0.40,0.40}{##1}}}
\expandafter\def\csname PY@tok@gh\endcsname{\let\PY@bf=\textbf\def\PY@tc##1{\textcolor[rgb]{0.00,0.00,0.50}{##1}}}
\expandafter\def\csname PY@tok@gu\endcsname{\let\PY@bf=\textbf\def\PY@tc##1{\textcolor[rgb]{0.50,0.00,0.50}{##1}}}
\expandafter\def\csname PY@tok@gd\endcsname{\def\PY@tc##1{\textcolor[rgb]{0.63,0.00,0.00}{##1}}}
\expandafter\def\csname PY@tok@gi\endcsname{\def\PY@tc##1{\textcolor[rgb]{0.00,0.63,0.00}{##1}}}
\expandafter\def\csname PY@tok@gr\endcsname{\def\PY@tc##1{\textcolor[rgb]{1.00,0.00,0.00}{##1}}}
\expandafter\def\csname PY@tok@ge\endcsname{\let\PY@it=\textit}
\expandafter\def\csname PY@tok@gs\endcsname{\let\PY@bf=\textbf}
\expandafter\def\csname PY@tok@gp\endcsname{\let\PY@bf=\textbf\def\PY@tc##1{\textcolor[rgb]{0.00,0.00,0.50}{##1}}}
\expandafter\def\csname PY@tok@go\endcsname{\def\PY@tc##1{\textcolor[rgb]{0.53,0.53,0.53}{##1}}}
\expandafter\def\csname PY@tok@gt\endcsname{\def\PY@tc##1{\textcolor[rgb]{0.00,0.27,0.87}{##1}}}
\expandafter\def\csname PY@tok@err\endcsname{\def\PY@bc##1{\setlength{\fboxsep}{0pt}\fcolorbox[rgb]{1.00,0.00,0.00}{1,1,1}{\strut ##1}}}
\expandafter\def\csname PY@tok@kc\endcsname{\let\PY@bf=\textbf\def\PY@tc##1{\textcolor[rgb]{0.00,0.50,0.00}{##1}}}
\expandafter\def\csname PY@tok@kd\endcsname{\let\PY@bf=\textbf\def\PY@tc##1{\textcolor[rgb]{0.00,0.50,0.00}{##1}}}
\expandafter\def\csname PY@tok@kn\endcsname{\let\PY@bf=\textbf\def\PY@tc##1{\textcolor[rgb]{0.00,0.50,0.00}{##1}}}
\expandafter\def\csname PY@tok@kr\endcsname{\let\PY@bf=\textbf\def\PY@tc##1{\textcolor[rgb]{0.00,0.50,0.00}{##1}}}
\expandafter\def\csname PY@tok@bp\endcsname{\def\PY@tc##1{\textcolor[rgb]{0.00,0.50,0.00}{##1}}}
\expandafter\def\csname PY@tok@fm\endcsname{\def\PY@tc##1{\textcolor[rgb]{0.00,0.00,1.00}{##1}}}
\expandafter\def\csname PY@tok@vc\endcsname{\def\PY@tc##1{\textcolor[rgb]{0.10,0.09,0.49}{##1}}}
\expandafter\def\csname PY@tok@vg\endcsname{\def\PY@tc##1{\textcolor[rgb]{0.10,0.09,0.49}{##1}}}
\expandafter\def\csname PY@tok@vi\endcsname{\def\PY@tc##1{\textcolor[rgb]{0.10,0.09,0.49}{##1}}}
\expandafter\def\csname PY@tok@vm\endcsname{\def\PY@tc##1{\textcolor[rgb]{0.10,0.09,0.49}{##1}}}
\expandafter\def\csname PY@tok@sa\endcsname{\def\PY@tc##1{\textcolor[rgb]{0.73,0.13,0.13}{##1}}}
\expandafter\def\csname PY@tok@sb\endcsname{\def\PY@tc##1{\textcolor[rgb]{0.73,0.13,0.13}{##1}}}
\expandafter\def\csname PY@tok@sc\endcsname{\def\PY@tc##1{\textcolor[rgb]{0.73,0.13,0.13}{##1}}}
\expandafter\def\csname PY@tok@dl\endcsname{\def\PY@tc##1{\textcolor[rgb]{0.73,0.13,0.13}{##1}}}
\expandafter\def\csname PY@tok@s2\endcsname{\def\PY@tc##1{\textcolor[rgb]{0.73,0.13,0.13}{##1}}}
\expandafter\def\csname PY@tok@sh\endcsname{\def\PY@tc##1{\textcolor[rgb]{0.73,0.13,0.13}{##1}}}
\expandafter\def\csname PY@tok@s1\endcsname{\def\PY@tc##1{\textcolor[rgb]{0.73,0.13,0.13}{##1}}}
\expandafter\def\csname PY@tok@mb\endcsname{\def\PY@tc##1{\textcolor[rgb]{0.40,0.40,0.40}{##1}}}
\expandafter\def\csname PY@tok@mf\endcsname{\def\PY@tc##1{\textcolor[rgb]{0.40,0.40,0.40}{##1}}}
\expandafter\def\csname PY@tok@mh\endcsname{\def\PY@tc##1{\textcolor[rgb]{0.40,0.40,0.40}{##1}}}
\expandafter\def\csname PY@tok@mi\endcsname{\def\PY@tc##1{\textcolor[rgb]{0.40,0.40,0.40}{##1}}}
\expandafter\def\csname PY@tok@il\endcsname{\def\PY@tc##1{\textcolor[rgb]{0.40,0.40,0.40}{##1}}}
\expandafter\def\csname PY@tok@mo\endcsname{\def\PY@tc##1{\textcolor[rgb]{0.40,0.40,0.40}{##1}}}
\expandafter\def\csname PY@tok@ch\endcsname{\let\PY@it=\textit\def\PY@tc##1{\textcolor[rgb]{0.25,0.50,0.50}{##1}}}
\expandafter\def\csname PY@tok@cm\endcsname{\let\PY@it=\textit\def\PY@tc##1{\textcolor[rgb]{0.25,0.50,0.50}{##1}}}
\expandafter\def\csname PY@tok@cpf\endcsname{\let\PY@it=\textit\def\PY@tc##1{\textcolor[rgb]{0.25,0.50,0.50}{##1}}}
\expandafter\def\csname PY@tok@c1\endcsname{\let\PY@it=\textit\def\PY@tc##1{\textcolor[rgb]{0.25,0.50,0.50}{##1}}}
\expandafter\def\csname PY@tok@cs\endcsname{\let\PY@it=\textit\def\PY@tc##1{\textcolor[rgb]{0.25,0.50,0.50}{##1}}}

\def\PYZbs{\char`\\}
\def\PYZus{\char`\_}
\def\PYZob{\char`\{}
\def\PYZcb{\char`\}}
\def\PYZca{\char`\^}
\def\PYZam{\char`\&}
\def\PYZlt{\char`\<}
\def\PYZgt{\char`\>}
\def\PYZsh{\char`\#}
\def\PYZpc{\char`\%}
\def\PYZdl{\char`\$}
\def\PYZhy{\char`\-}
\def\PYZsq{\char`\'}
\def\PYZdq{\char`\"}
\def\PYZti{\char`\~}
% for compatibility with earlier versions
\def\PYZat{@}
\def\PYZlb{[}
\def\PYZrb{]}
\makeatother


    % Exact colors from NB
    \definecolor{incolor}{rgb}{0.0, 0.0, 0.5}
    \definecolor{outcolor}{rgb}{0.545, 0.0, 0.0}



    
    % Prevent overflowing lines due to hard-to-break entities
    \sloppy 
    % Setup hyperref package
    \hypersetup{
      breaklinks=true,  % so long urls are correctly broken across lines
      colorlinks=true,
      urlcolor=urlcolor,
      linkcolor=linkcolor,
      citecolor=citecolor,
      }
    % Slightly bigger margins than the latex defaults
    
    \geometry{verbose,tmargin=1in,bmargin=1in,lmargin=1in,rmargin=1in}
    
    

    \begin{document}
    
    
    \maketitle
    
    

    
    \hypertarget{notebook-relacionado-con-el-pronuxf3stico-del-precio-de-venta-de-los-productos-agricolas.}{%
\section{Notebook relacionado con el pronóstico del precio de venta de
los productos
agricolas.}\label{notebook-relacionado-con-el-pronuxf3stico-del-precio-de-venta-de-los-productos-agricolas.}}

    Teniendo en cuenta que las dos funciones objetivo del proyecto de
investigación titulado: \textbf{Modelo de optimización multiobjetivo
para la programación de la producción agrícola a pequeña escala en
Santander, Colombia}, tienen en consideración el parámetro de precios de
venta de cada kilo de producto agrícola recogido durante el horizonte de
planeación (e.g.~Objetivo 1: Maximizar ingresos por ventas, Objetivo 2:
Minimizar riesgo financiero), y que la prueba del modelo se realizará en
un contexto local (Departamento de Santander); se propone la estimación
de los precios a partir de la formulación de modelos auto regresivo. Los
modelos son desarrollados a partir del análisis histórico de los precios
de venta por kilogramo de cada producto en la central de abastos de
Bucaramanga, en un periodo comprendido entre la primera semana de enero
de 2013 hasta la última semana de diciembre de 2017.

    El Notebook está estructurado de tal manera que sea posible hacer un
seguimiento a la construción de los modelos de pronósticos. Para ello,
se divide en las siguientes secciones:

Section \ref{seccion201}

Section \ref{seccion202}

Section \ref{seccion203}

Section \ref{seccion204}

Section \ref{seccion205}

Section \ref{seccion206}

    \hypertarget{adquisiciuxf3n-de-los-datos-y-depuraciuxf3n}{%
\section{ 1. Adquisición de los datos y
depuración}\label{adquisiciuxf3n-de-los-datos-y-depuraciuxf3n}}

    En la primera sección se cargan los paquetes necesarios para trabajar
los modelos de pronósticos:

    \begin{Verbatim}[commandchars=\\\{\}]
{\color{incolor}In [{\color{incolor}6}]:} \PY{k+kn}{library}\PY{p}{(}randtests\PY{p}{)}\PY{c+c1}{\PYZsh{}Evaluar pruebas de estacionariedad}
        \PY{k+kn}{library}\PY{p}{(}tseries\PY{p}{)}  \PY{c+c1}{\PYZsh{}Evaluar pruebas de estacionariedad}
        \PY{k+kn}{library}\PY{p}{(}urca\PY{p}{)}     \PY{c+c1}{\PYZsh{}Evaluar la presencia de raices unitarias}
        \PY{k+kn}{library}\PY{p}{(}Amelia\PY{p}{)}
        \PY{k+kn}{library}\PY{p}{(}lmtest\PY{p}{)}   \PY{c+c1}{\PYZsh{}Generar modelos de regresión lineal}
        \PY{k+kn}{library}\PY{p}{(}mice\PY{p}{)}
        \PY{k+kn}{library}\PY{p}{(}readr\PY{p}{)}    \PY{c+c1}{\PYZsh{}Cargar archivos csv}
        \PY{k+kn}{library}\PY{p}{(}ggplot2\PY{p}{)}  \PY{c+c1}{\PYZsh{}Construcción de gráficas}
        \PY{k+kn}{library}\PY{p}{(}forecast\PY{p}{)} \PY{c+c1}{\PYZsh{}Modelos de pronóstico}
        \PY{k+kn}{library}\PY{p}{(}reshape2\PY{p}{)} \PY{c+c1}{\PYZsh{}Construcción de gráficas}
        \PY{k+kn}{library}\PY{p}{(}VIM\PY{p}{)}      \PY{c+c1}{\PYZsh{}IMputación de datos}
\end{Verbatim}


    Ahora es cargada la base de datos de precios y se observan los campos
que la componen:

    \begin{Verbatim}[commandchars=\\\{\}]
{\color{incolor}In [{\color{incolor}1}]:} DATA \PY{o}{\PYZlt{}\PYZhy{}} read.csv\PY{p}{(}file\PY{o}{=}\PY{l+s}{\PYZdq{}}\PY{l+s}{DATA\PYZus{}BUCARA.csv\PYZdq{}}\PY{p}{,} header\PY{o}{=}\PY{k+kc}{TRUE}\PY{p}{,} sep\PY{o}{=}\PY{l+s}{\PYZdq{}}\PY{l+s}{;\PYZdq{}}\PY{p}{)} \PY{c+c1}{\PYZsh{}carga desde archivo .csv ubicado en el directorio raiz}
        DATA \PY{o}{\PYZlt{}\PYZhy{}} \PY{k+kp}{as.data.frame}\PY{p}{(}DATA\PY{p}{)} \PY{c+c1}{\PYZsh{}transformación de los datos a formato Data Frame}
\end{Verbatim}


    \hypertarget{descripciuxf3n-del-dataset}{%
\subsection{Descripción del Dataset}\label{descripciuxf3n-del-dataset}}

    \begin{Verbatim}[commandchars=\\\{\}]
{\color{incolor}In [{\color{incolor}70}]:} \PY{k+kp}{print}\PY{p}{(}\PY{l+s}{\PYZdq{}}\PY{l+s}{Visualización de los 5 primeros datos\PYZdq{}}\PY{p}{)}
         \PY{k+kp}{head}\PY{p}{(}DATA\PY{p}{)}
         \PY{k+kp}{print}\PY{p}{(}\PY{l+s}{\PYZdq{}}\PY{l+s}{Características de los datos\PYZdq{}}\PY{p}{)}
         \PY{k+kp}{print}\PY{p}{(}str\PY{p}{(}DATA\PY{p}{)}\PY{p}{)}
         \PY{k+kp}{print}\PY{p}{(}\PY{l+s}{\PYZdq{}}\PY{l+s}{Resumen descriptivo de los datos\PYZdq{}}\PY{p}{)}
         \PY{k+kp}{summary}\PY{p}{(}DATA\PY{p}{)}
         \PY{k+kp}{print}\PY{p}{(}\PY{l+s}{\PYZdq{}}\PY{l+s}{Visualización de los datos perdidos\PYZdq{}}\PY{p}{)}
         \PY{c+c1}{\PYZsh{}Con el propósito de identificar si existen datos perdidos (una aproximación visual), se adapta una función creada por Nicholas Tierney https://njtierney.github.io/}
         
         \PY{c+c1}{\PYZsh{}Inicio de la función}
         ggplot\PYZus{}missing \PY{o}{\PYZlt{}\PYZhy{}} \PY{k+kr}{function}\PY{p}{(}x\PY{p}{)}\PY{p}{\PYZob{}}
           
           x \PY{o}{\PYZpc{}\PYZgt{}\PYZpc{}} 
             is.na \PY{o}{\PYZpc{}\PYZgt{}\PYZpc{}}
             melt \PY{o}{\PYZpc{}\PYZgt{}\PYZpc{}}
             ggplot\PY{p}{(}data \PY{o}{=} \PY{l+m}{.}\PY{p}{,}
                    aes\PY{p}{(}x \PY{o}{=} Var2\PY{p}{,}
                        y \PY{o}{=} Var1\PY{p}{)}\PY{p}{)} \PY{o}{+}
             geom\PYZus{}raster\PY{p}{(}aes\PY{p}{(}fill \PY{o}{=} value\PY{p}{)}\PY{p}{)} \PY{o}{+}
             scale\PYZus{}fill\PYZus{}grey\PY{p}{(}name \PY{o}{=} \PY{l+s}{\PYZdq{}}\PY{l+s}{\PYZdq{}}\PY{p}{,}
                             labels \PY{o}{=} \PY{k+kt}{c}\PY{p}{(}\PY{l+s}{\PYZdq{}}\PY{l+s}{Presentes\PYZdq{}}\PY{p}{,}\PY{l+s}{\PYZdq{}}\PY{l+s}{Perdidos\PYZdq{}}\PY{p}{)}\PY{p}{)} \PY{o}{+}
             theme\PYZus{}minimal\PY{p}{(}\PY{p}{)} \PY{o}{+} 
             theme\PY{p}{(}axis.text.x  \PY{o}{=} element\PYZus{}text\PY{p}{(}angle\PY{o}{=}\PY{l+m}{45}\PY{p}{,} vjust\PY{o}{=}\PY{l+m}{0.5}\PY{p}{)}\PY{p}{)} \PY{o}{+} 
             labs\PY{p}{(}x \PY{o}{=} \PY{l+s}{\PYZdq{}}\PY{l+s}{Variables en el Dataset\PYZdq{}}\PY{p}{,}
                  y \PY{o}{=} \PY{l+s}{\PYZdq{}}\PY{l+s}{Filas / observaciones\PYZdq{}}\PY{p}{)}
         \PY{p}{\PYZcb{}}
         \PY{c+c1}{\PYZsh{}Fin de la Función}
         ggplot\PYZus{}missing\PY{p}{(}DATA\PY{p}{)}
\end{Verbatim}


    \begin{Verbatim}[commandchars=\\\{\}]
[1] "Visualización de los 5 primeros datos"

    \end{Verbatim}

    \begin{tabular}{r|llllllllllllllllllllllllll}
 Fecha & Ahuyama & Ajo & Arverja & Cebolla.Cabezona & Cebolla.junca & Cilantro & Cebada & Maiz & Frijol & ... & Banano & Lulo & Mora & Papaya & Patilla & Pinna & Arracacha & Papa & Plátano & Yuca\\
\hline
	 04/01/2013 & 676        & 3243       & 2158       & 933        & 920        & 5569       & 1760       & 2040       & 1481       & ...        &  722       & 1749       & 1497       & 1063       & 600        & 625        & 1130       & 814        & 1291       & 1240      \\
	 11/01/2013 & 652        & 3268       & 1921       & 823        & 665        & 7317       & 1680       & 1920       & 1578       & ...        &  819       & 1732       & 1638       & 1093       & 600        & 632        & 1149       & 799        & 1163       & 1108      \\
	 12/01/2013 & 614        & 3298       & 1761       & 718        & 724        & 6250       & 1840       & 1947       & 1630       & ...        &  762       & 1576       & 1843       & 1041       & 607        & 579        & 1067       & 781        & 1103       &  856      \\
	 19/01/2013 & 636        & 3384       & 1750       & 700        & 604        & 3854       & 1653       & 1947       & 1599       & ...        &  799       & 1650       & 1970       & 1034       & 663        & 635        & 1075       & 783        & 1158       &  836      \\
	 26/01/2013 & 619        & 3400       & 1905       & 625        & 646        & 4333       & 1653       & 1920       & 1451       & ...        &  967       & 1599       & 2067       & 1073       & 698        & 626        & 1013       & 787        & 1075       &  846      \\
	 02/02/2013 & 620        & 3332       & 1892       & 630        & 657        & 6833       & 1707       & 1893       & 1719       & ...        & 1152       & 1881       & 2304       & 1200       & 727        & 716        &  979       & 735        & 1140       &  833      \\
\end{tabular}


    
    \begin{Verbatim}[commandchars=\\\{\}]
[1] "Características de los datos"
'data.frame':	261 obs. of  26 variables:
 \$ Fecha           : Factor w/ 260 levels "01/02/2014","01/03/2014",..: 27 88 99 157 217 10 73 133 192 11 {\ldots}
 \$ Ahuyama         : int  676 652 614 636 619 620 632 600 581 600 {\ldots}
 \$ Ajo             : int  3243 3268 3298 3384 3400 3332 3384 3393 3450 3520 {\ldots}
 \$ Arverja         : int  2158 1921 1761 1750 1905 1892 1711 1893 1975 3520 {\ldots}
 \$ Cebolla.Cabezona: int  933 823 718 700 625 630 633 575 506 575 {\ldots}
 \$ Cebolla.junca   : int  920 665 724 604 646 657 622 668 623 689 {\ldots}
 \$ Cilantro        : int  5569 7317 6250 3854 4333 6833 7719 9619 NA NA {\ldots}
 \$ Cebada          : int  1760 1680 1840 1653 1653 1707 1653 1667 1808 1747 {\ldots}
 \$ Maiz            : int  2040 1920 1947 1947 1920 1893 1960 1920 2028 1960 {\ldots}
 \$ Frijol          : int  1481 1578 1630 1599 1451 1719 1667 1830 1773 2368 {\ldots}
 \$ Habichuela      : int  961 1228 1484 1624 1272 1391 1604 1951 1820 1624 {\ldots}
 \$ Lechuga         : int  1186 1146 1029 1013 876 904 919 1098 1172 1496 {\ldots}
 \$ Perejil         : int  5986 5979 5823 4750 4000 4698 4250 5177 NA 6156 {\ldots}
 \$ Pimenton        : int  1533 1106 1029 1274 1393 1910 2234 2378 1843 1880 {\ldots}
 \$ Repollo.blanco  : int  507 510 513 530 488 492 480 480 481 477 {\ldots}
 \$ Tomate          : int  1085 913 1167 1324 1057 1051 1011 1152 826 958 {\ldots}
 \$ Banano          : int  722 819 762 799 967 1152 1156 1177 1227 1163 {\ldots}
 \$ Lulo            : int  1749 1732 1576 1650 1599 1881 1992 1965 1978 2189 {\ldots}
 \$ Mora            : int  1497 1638 1843 1970 2067 2304 2527 2937 2480 1703 {\ldots}
 \$ Papaya          : int  1063 1093 1041 1034 1073 1200 1206 1295 1249 1200 {\ldots}
 \$ Patilla         : int  600 600 607 663 698 727 719 703 688 625 {\ldots}
 \$ Pinna           : int  625 632 579 635 626 716 684 702 686 730 {\ldots}
 \$ Arracacha       : int  1130 1149 1067 1075 1013 979 1048 925 1016 1089 {\ldots}
 \$ Papa            : int  814 799 781 783 787 735 723 719 715 818 {\ldots}
 \$ Plátano         : int  1291 1163 1103 1158 1075 1140 1180 1520 1631 1680 {\ldots}
 \$ Yuca            : int  1240 1108 856 836 846 833 866 804 878 847 {\ldots}
NULL
[1] "Resumen descriptivo de los datos"

    \end{Verbatim}

    
    \begin{verbatim}
        Fecha        Ahuyama            Ajo           Arverja    
 21/12/2013:  2   Min.   : 422.0   Min.   : 2341   Min.   :1163  
 01/02/2014:  1   1st Qu.: 568.0   1st Qu.: 3298   1st Qu.:2160  
 01/03/2014:  1   Median : 660.0   Median : 5000   Median :2660  
 01/04/2017:  1   Mean   : 700.4   Mean   : 5497   Mean   :2933  
 01/05/2015:  1   3rd Qu.: 800.0   3rd Qu.: 7921   3rd Qu.:3350  
 01/07/2017:  1   Max.   :1354.0   Max.   :11000   Max.   :8033  
 (Other)   :254                                    NA's   :1     
 Cebolla.Cabezona Cebolla.junca     Cilantro         Cebada          Maiz     
 Min.   : 386     Min.   : 418   Min.   : 2264   Min.   :1603   Min.   :1769  
 1st Qu.: 808     1st Qu.: 810   1st Qu.: 4838   1st Qu.:1656   1st Qu.:1907  
 Median :1225     Median :1019   Median : 6000   Median :1753   Median :1960  
 Mean   :1301     Mean   :1113   Mean   : 6344   Mean   :1771   Mean   :1994  
 3rd Qu.:1657     3rd Qu.:1245   3rd Qu.: 7500   3rd Qu.:1893   3rd Qu.:2080  
 Max.   :3350     Max.   :3239   Max.   :15146   Max.   :2037   Max.   :2272  
                                 NA's   :3                                    
     Frijol       Habichuela      Lechuga        Perejil         Pimenton   
 Min.   :1066   Min.   : 553   Min.   : 667   Min.   : 1769   Min.   : 988  
 1st Qu.:1719   1st Qu.:1078   1st Qu.: 957   1st Qu.: 4000   1st Qu.:1386  
 Median :2077   Median :1367   Median :1025   Median : 4984   Median :1781  
 Mean   :2166   Mean   :1441   Mean   :1086   Mean   : 5392   Mean   :1911  
 3rd Qu.:2477   3rd Qu.:1724   3rd Qu.:1159   3rd Qu.: 6000   3rd Qu.:2188  
 Max.   :4013   Max.   :3569   Max.   :2427   Max.   :13000   Max.   :4104  
                                              NA's   :1                     
 Repollo.blanco     Tomate         Banano          Lulo           Mora     
 Min.   : 400   Min.   : 674   Min.   : 722   Min.   :1549   Min.   :1365  
 1st Qu.: 510   1st Qu.:1144   1st Qu.:1124   1st Qu.:2205   1st Qu.:1980  
 Median : 638   Median :1432   Median :1250   Median :2521   Median :2286  
 Mean   : 682   Mean   :1604   Mean   :1334   Mean   :2636   Mean   :2433  
 3rd Qu.: 752   3rd Qu.:1982   3rd Qu.:1600   3rd Qu.:2945   3rd Qu.:2750  
 Max.   :1440   Max.   :3442   Max.   :1925   Max.   :4575   Max.   :5650  
                                                                           
     Papaya        Patilla           Pinna         Arracacha         Papa     
 Min.   : 850   Min.   : 575.0   Min.   :460.0   Min.   : 572   Min.   : 627  
 1st Qu.:1098   1st Qu.: 740.0   1st Qu.:631.0   1st Qu.: 780   1st Qu.: 900  
 Median :1255   Median : 855.0   Median :676.0   Median :1024   Median :1000  
 Mean   :1280   Mean   : 855.8   Mean   :665.4   Mean   :1407   Mean   :1089  
 3rd Qu.:1423   3rd Qu.: 950.0   3rd Qu.:716.0   3rd Qu.:1755   3rd Qu.:1242  
 Max.   :2440   Max.   :1211.0   Max.   :862.0   Max.   :4107   Max.   :2000  
                                                                NA's   :15    
    Plátano          Yuca       
 Min.   : 992   Min.   : 596.0  
 1st Qu.:1182   1st Qu.: 714.0  
 Median :1328   Median : 833.0  
 Mean   :1419   Mean   : 930.2  
 3rd Qu.:1562   3rd Qu.:1143.0  
 Max.   :2448   Max.   :1664.0  
                                
    \end{verbatim}

    
    \begin{Verbatim}[commandchars=\\\{\}]
[1] "Visualización de los datos perdidos"

    \end{Verbatim}

    
    
    \begin{center}
    \adjustimage{max size={0.9\linewidth}{0.9\paperheight}}{output_9_6.png}
    \end{center}
    { \hspace*{\fill} \\}
    
    \hypertarget{imputaciuxf3n-del-dataset}{%
\subsection{Imputación del Dataset}\label{imputaciuxf3n-del-dataset}}

    Teniendo en cuenta que no están presentes todos los datos en el Dataset
de precios, se procede a realizar la imputación del mismo. Para la
imputación se realiza un cálculo previo de la cantidad de datos
perdidos, con dicho valor y teniendo en cuenta la dispersión de los
datos observadas en la Figura \textbf{Visualización de datos perdidos},
se selecciona la imputación de promedio de k vecinos, con 5 vecinos
\textbf{BUSCAR CITA}

    \begin{Verbatim}[commandchars=\\\{\}]
{\color{incolor}In [{\color{incolor}71}]:} \PY{c+c1}{\PYZsh{}Cálculo de datos perdidos}
         \PY{k+kp}{print}\PY{p}{(}\PY{l+s}{\PYZdq{}}\PY{l+s}{La cantidad de datos perdido por variable antes de imputar es\PYZdq{}}\PY{p}{)}
         \PY{k+kp}{colSums}\PY{p}{(}is.na \PY{p}{(}DATA\PY{p}{)}\PY{p}{)}
         \PY{c+c1}{\PYZsh{}Imputación mediante k vecinos}
         \PY{c+c1}{\PYZsh{}Se aplica la función kNN (K Nearest Neighbour Imputation) y se guarda como una nueva variable}
         DATA\PYZus{}IMP\PY{o}{\PYZlt{}\PYZhy{}}kNN\PY{p}{(}DATA\PY{p}{,} variable \PY{o}{=} \PY{k+kp}{colnames}\PY{p}{(}DATA\PY{p}{)}\PY{p}{,} metric \PY{o}{=} \PY{k+kc}{NULL}\PY{p}{,} k \PY{o}{=} \PY{l+m}{5}\PY{p}{,}
           dist\PYZus{}var \PY{o}{=} \PY{k+kp}{colnames}\PY{p}{(}DATA\PY{p}{)}\PY{p}{,} weights \PY{o}{=} \PY{k+kc}{NULL}\PY{p}{,} numFun \PY{o}{=} median\PY{p}{,}
           catFun \PY{o}{=} maxCat\PY{p}{,} makeNA \PY{o}{=} \PY{k+kc}{NULL}\PY{p}{,} NAcond \PY{o}{=} \PY{k+kc}{NULL}\PY{p}{,} impNA \PY{o}{=} \PY{k+kc}{TRUE}\PY{p}{,}
           donorcond \PY{o}{=} \PY{k+kc}{NULL}\PY{p}{,} mixed \PY{o}{=} \PY{k+kt}{vector}\PY{p}{(}\PY{p}{)}\PY{p}{,} mixed.constant \PY{o}{=} \PY{k+kc}{NULL}\PY{p}{,}
           trace \PY{o}{=} \PY{k+kc}{FALSE}\PY{p}{,} imp\PYZus{}var \PY{o}{=} \PY{k+kc}{TRUE}\PY{p}{,} imp\PYZus{}suffix \PY{o}{=} \PY{l+s}{\PYZdq{}}\PY{l+s}{imp\PYZdq{}}\PY{p}{,} addRandom \PY{o}{=} \PY{k+kc}{FALSE}\PY{p}{,}
           useImputedDist \PY{o}{=} \PY{k+kc}{TRUE}\PY{p}{,} weightDist \PY{o}{=} \PY{k+kc}{FALSE}\PY{p}{)}
         \PY{c+c1}{\PYZsh{}Los datos son actualizados en el Dataframe}
         DATA\PY{o}{\PYZlt{}\PYZhy{}}\PY{k+kp}{as.data.frame}\PY{p}{(}DATA\PYZus{}IMP\PY{p}{[}\PY{p}{,}\PY{l+m}{1}\PY{o}{:}\PY{k+kp}{length}\PY{p}{(}DATA\PY{p}{)}\PY{p}{]}\PY{p}{)}
\end{Verbatim}


    \begin{Verbatim}[commandchars=\\\{\}]
[1] "La cantidad de datos perdido por variable antes de imputar es"

    \end{Verbatim}

    \begin{description*}
\item[Fecha] 0
\item[Ahuyama] 0
\item[Ajo] 0
\item[Arverja] 1
\item[Cebolla.Cabezona] 0
\item[Cebolla.junca] 0
\item[Cilantro] 3
\item[Cebada] 0
\item[Maiz] 0
\item[Frijol] 0
\item[Habichuela] 0
\item[Lechuga] 0
\item[Perejil] 1
\item[Pimenton] 0
\item[Repollo.blanco] 0
\item[Tomate] 0
\item[Banano] 0
\item[Lulo] 0
\item[Mora] 0
\item[Papaya] 0
\item[Patilla] 0
\item[Pinna] 0
\item[Arracacha] 0
\item[Papa] 15
\item[Plátano] 0
\item[Yuca] 0
\end{description*}


    
    Finalmente, las series históricas se presentan visualmente a
continuación:

    \begin{Verbatim}[commandchars=\\\{\}]
{\color{incolor}In [{\color{incolor}72}]:} \PY{c+c1}{\PYZsh{}Primero se construye una función para graficar todas las series de manera independiente utilziando un bucle}
         
         \PY{c+c1}{\PYZsh{}Creación de la función}
         GRAFICAR \PY{o}{\PYZlt{}\PYZhy{}} \PY{k+kr}{function}\PY{p}{(}x\PY{p}{,} na.rm \PY{o}{=} \PY{k+kc}{TRUE}\PY{p}{,} \PY{k+kc}{...}\PY{p}{)}
             \PY{p}{\PYZob{}}
             nm \PY{o}{\PYZlt{}\PYZhy{}} \PY{k+kp}{names}\PY{p}{(}x\PY{p}{)}
               \PY{k+kr}{for} \PY{p}{(}i \PY{k+kr}{in} \PY{k+kp}{seq\PYZus{}along}\PY{p}{(}nm\PY{p}{[}\PY{l+m}{2}\PY{o}{:}\PY{k+kp}{length}\PY{p}{(}x\PY{p}{[}\PY{l+m}{1}\PY{p}{,}\PY{p}{]}\PY{p}{)}\PY{p}{]}\PY{p}{)}\PY{p}{)} 
                 \PY{p}{\PYZob{}}
                   
                 plots \PY{o}{\PYZlt{}\PYZhy{}}ggplot\PY{p}{(}x\PY{p}{,}aes\PY{p}{(}\PY{k+kp}{as.Date.character}\PY{p}{(}x\PY{p}{[}\PY{p}{,}\PY{l+m}{1}\PY{p}{]}\PY{p}{,}\PY{l+s}{\PYZdq{}}\PY{l+s}{\PYZpc{}m/\PYZpc{}d/\PYZpc{}Y\PYZdq{}}\PY{p}{)}\PY{p}{,} x\PY{p}{[}\PY{p}{,}\PY{p}{(}i\PY{l+m}{+1}\PY{p}{)}\PY{p}{]}\PY{p}{)}\PY{p}{)} \PY{o}{+} geom\PYZus{}line\PY{p}{(}\PY{p}{)}
                   plots\PY{o}{\PYZlt{}\PYZhy{}}plots\PY{o}{+}labs\PY{p}{(}x \PY{o}{=} \PY{l+s}{\PYZdq{}}\PY{l+s}{Fecha\PYZdq{}}\PY{p}{)}\PY{o}{+}labs\PY{p}{(}y \PY{o}{=} \PY{l+s}{\PYZdq{}}\PY{l+s}{Pesos colombianos [COP]\PYZdq{}}\PY{p}{)}\PY{o}{+}labs\PY{p}{(}title \PY{o}{=} \PY{l+s}{\PYZdq{}}\PY{l+s}{Precio historico para el producto:\PYZdq{}}\PY{p}{,} subtitle \PY{o}{=} nm\PY{p}{[}i\PY{l+m}{+1}\PY{p}{]}\PY{p}{)}\PY{o}{+}scale\PYZus{}x\PYZus{}date\PY{p}{(}date\PYZus{}labels \PY{o}{=} \PY{l+s}{\PYZdq{}}\PY{l+s}{\PYZpc{}b \PYZpc{}Y\PYZdq{}}\PY{p}{)}
                 \PY{k+kp}{print}\PY{p}{(}plots\PY{p}{)}
         
                   \PY{c+c1}{\PYZsh{}Opcional si desean guardar la imagen}
                   \PY{c+c1}{\PYZsh{}ggplot(x,aes(x[,1], x[,(i)])) + geom\PYZus{}line()}
                   \PY{c+c1}{\PYZsh{}ggsave(plots,filename=paste(\PYZdq{}00 \PYZdq{},nm[i],\PYZdq{}.png\PYZdq{},sep=\PYZdq{}\PYZdq{}))}
                 \PY{p}{\PYZcb{}}
             \PY{p}{\PYZcb{}}
         \PY{c+c1}{\PYZsh{}Fin de la función}
         
         \PY{c+c1}{\PYZsh{}Se ejecuta la función}
         GRAFICAR\PY{p}{(}DATA\PY{p}{)}
\end{Verbatim}


    \begin{Verbatim}[commandchars=\\\{\}]
Warning message:
"Removed 155 rows containing missing values (geom\_path)."Warning message:
"Removed 155 rows containing missing values (geom\_path)."
    \end{Verbatim}

    \begin{center}
    \adjustimage{max size={0.9\linewidth}{0.9\paperheight}}{output_14_1.png}
    \end{center}
    { \hspace*{\fill} \\}
    
    \begin{Verbatim}[commandchars=\\\{\}]
Warning message:
"Removed 155 rows containing missing values (geom\_path)."
    \end{Verbatim}

    \begin{center}
    \adjustimage{max size={0.9\linewidth}{0.9\paperheight}}{output_14_3.png}
    \end{center}
    { \hspace*{\fill} \\}
    
    \begin{Verbatim}[commandchars=\\\{\}]
Warning message:
"Removed 155 rows containing missing values (geom\_path)."
    \end{Verbatim}

    \begin{center}
    \adjustimage{max size={0.9\linewidth}{0.9\paperheight}}{output_14_5.png}
    \end{center}
    { \hspace*{\fill} \\}
    
    \begin{Verbatim}[commandchars=\\\{\}]
Warning message:
"Removed 155 rows containing missing values (geom\_path)."
    \end{Verbatim}

    \begin{center}
    \adjustimage{max size={0.9\linewidth}{0.9\paperheight}}{output_14_7.png}
    \end{center}
    { \hspace*{\fill} \\}
    
    \begin{Verbatim}[commandchars=\\\{\}]
Warning message:
"Removed 155 rows containing missing values (geom\_path)."
    \end{Verbatim}

    \begin{center}
    \adjustimage{max size={0.9\linewidth}{0.9\paperheight}}{output_14_9.png}
    \end{center}
    { \hspace*{\fill} \\}
    
    \begin{Verbatim}[commandchars=\\\{\}]
Warning message:
"Removed 155 rows containing missing values (geom\_path)."
    \end{Verbatim}

    \begin{center}
    \adjustimage{max size={0.9\linewidth}{0.9\paperheight}}{output_14_11.png}
    \end{center}
    { \hspace*{\fill} \\}
    
    \begin{Verbatim}[commandchars=\\\{\}]
Warning message:
"Removed 155 rows containing missing values (geom\_path)."
    \end{Verbatim}

    \begin{center}
    \adjustimage{max size={0.9\linewidth}{0.9\paperheight}}{output_14_13.png}
    \end{center}
    { \hspace*{\fill} \\}
    
    \begin{Verbatim}[commandchars=\\\{\}]
Warning message:
"Removed 155 rows containing missing values (geom\_path)."
    \end{Verbatim}

    \begin{center}
    \adjustimage{max size={0.9\linewidth}{0.9\paperheight}}{output_14_15.png}
    \end{center}
    { \hspace*{\fill} \\}
    
    \begin{Verbatim}[commandchars=\\\{\}]
Warning message:
"Removed 155 rows containing missing values (geom\_path)."
    \end{Verbatim}

    \begin{center}
    \adjustimage{max size={0.9\linewidth}{0.9\paperheight}}{output_14_17.png}
    \end{center}
    { \hspace*{\fill} \\}
    
    \begin{Verbatim}[commandchars=\\\{\}]
Warning message:
"Removed 155 rows containing missing values (geom\_path)."
    \end{Verbatim}

    \begin{center}
    \adjustimage{max size={0.9\linewidth}{0.9\paperheight}}{output_14_19.png}
    \end{center}
    { \hspace*{\fill} \\}
    
    \begin{Verbatim}[commandchars=\\\{\}]
Warning message:
"Removed 155 rows containing missing values (geom\_path)."
    \end{Verbatim}

    \begin{center}
    \adjustimage{max size={0.9\linewidth}{0.9\paperheight}}{output_14_21.png}
    \end{center}
    { \hspace*{\fill} \\}
    
    \begin{Verbatim}[commandchars=\\\{\}]
Warning message:
"Removed 155 rows containing missing values (geom\_path)."
    \end{Verbatim}

    \begin{center}
    \adjustimage{max size={0.9\linewidth}{0.9\paperheight}}{output_14_23.png}
    \end{center}
    { \hspace*{\fill} \\}
    
    \begin{Verbatim}[commandchars=\\\{\}]
Warning message:
"Removed 155 rows containing missing values (geom\_path)."
    \end{Verbatim}

    \begin{center}
    \adjustimage{max size={0.9\linewidth}{0.9\paperheight}}{output_14_25.png}
    \end{center}
    { \hspace*{\fill} \\}
    
    \begin{Verbatim}[commandchars=\\\{\}]
Warning message:
"Removed 155 rows containing missing values (geom\_path)."
    \end{Verbatim}

    \begin{center}
    \adjustimage{max size={0.9\linewidth}{0.9\paperheight}}{output_14_27.png}
    \end{center}
    { \hspace*{\fill} \\}
    
    \begin{Verbatim}[commandchars=\\\{\}]
Warning message:
"Removed 155 rows containing missing values (geom\_path)."
    \end{Verbatim}

    \begin{center}
    \adjustimage{max size={0.9\linewidth}{0.9\paperheight}}{output_14_29.png}
    \end{center}
    { \hspace*{\fill} \\}
    
    \begin{Verbatim}[commandchars=\\\{\}]
Warning message:
"Removed 155 rows containing missing values (geom\_path)."
    \end{Verbatim}

    \begin{center}
    \adjustimage{max size={0.9\linewidth}{0.9\paperheight}}{output_14_31.png}
    \end{center}
    { \hspace*{\fill} \\}
    
    \begin{Verbatim}[commandchars=\\\{\}]
Warning message:
"Removed 155 rows containing missing values (geom\_path)."
    \end{Verbatim}

    \begin{center}
    \adjustimage{max size={0.9\linewidth}{0.9\paperheight}}{output_14_33.png}
    \end{center}
    { \hspace*{\fill} \\}
    
    \begin{Verbatim}[commandchars=\\\{\}]
Warning message:
"Removed 155 rows containing missing values (geom\_path)."
    \end{Verbatim}

    \begin{center}
    \adjustimage{max size={0.9\linewidth}{0.9\paperheight}}{output_14_35.png}
    \end{center}
    { \hspace*{\fill} \\}
    
    \begin{Verbatim}[commandchars=\\\{\}]
Warning message:
"Removed 155 rows containing missing values (geom\_path)."
    \end{Verbatim}

    \begin{center}
    \adjustimage{max size={0.9\linewidth}{0.9\paperheight}}{output_14_37.png}
    \end{center}
    { \hspace*{\fill} \\}
    
    \begin{Verbatim}[commandchars=\\\{\}]
Warning message:
"Removed 155 rows containing missing values (geom\_path)."
    \end{Verbatim}

    \begin{center}
    \adjustimage{max size={0.9\linewidth}{0.9\paperheight}}{output_14_39.png}
    \end{center}
    { \hspace*{\fill} \\}
    
    \begin{Verbatim}[commandchars=\\\{\}]
Warning message:
"Removed 155 rows containing missing values (geom\_path)."
    \end{Verbatim}

    \begin{center}
    \adjustimage{max size={0.9\linewidth}{0.9\paperheight}}{output_14_41.png}
    \end{center}
    { \hspace*{\fill} \\}
    
    \begin{Verbatim}[commandchars=\\\{\}]
Warning message:
"Removed 155 rows containing missing values (geom\_path)."
    \end{Verbatim}

    \begin{center}
    \adjustimage{max size={0.9\linewidth}{0.9\paperheight}}{output_14_43.png}
    \end{center}
    { \hspace*{\fill} \\}
    
    \begin{Verbatim}[commandchars=\\\{\}]
Warning message:
"Removed 155 rows containing missing values (geom\_path)."
    \end{Verbatim}

    \begin{center}
    \adjustimage{max size={0.9\linewidth}{0.9\paperheight}}{output_14_45.png}
    \end{center}
    { \hspace*{\fill} \\}
    
    \begin{Verbatim}[commandchars=\\\{\}]
Warning message:
"Removed 155 rows containing missing values (geom\_path)."
    \end{Verbatim}

    \begin{center}
    \adjustimage{max size={0.9\linewidth}{0.9\paperheight}}{output_14_47.png}
    \end{center}
    { \hspace*{\fill} \\}
    
    \begin{center}
    \adjustimage{max size={0.9\linewidth}{0.9\paperheight}}{output_14_48.png}
    \end{center}
    { \hspace*{\fill} \\}
    
    \hypertarget{verificaciuxf3n-de-los-supuestos-de-estacionariedad}{%
\section{ 2. Verificación de los supuestos de
Estacionariedad}\label{verificaciuxf3n-de-los-supuestos-de-estacionariedad}}

    Una vez construído el Dataset, se aplican diversas pruebas con el
propósito de determinan si las series son o no estacionarias.

    \hypertarget{descripciuxf3n-de-las-pruebas}{%
\subsection{2.1 Descripción de las
pruebas}\label{descripciuxf3n-de-las-pruebas}}

    \hypertarget{prueba-de-raiz-unitaria}{%
\subsubsection{2.1.1 Prueba de raiz
Unitaria}\label{prueba-de-raiz-unitaria}}

    Las primeras tres pruebas aplicadas durante el presente trabajo
pertenecen a la familia de las raíces unitarias, las cuales según citan
Nisar y Hanif: (2012, p.~418) ``pueden ser usadas para evaluar la
eficiencia de los mercados, puesto que dicho comportamiento demanda
aleatoriedad (i.e.~serie no estacionaria) en los precios (Hassan,
Abdullah, \& Shah, 2007)''. Éstas son evaluadas utilizando el paquete
``urca''.

\textbf{1. Raíz unitaria con deriva y con tendencia} La primera prueba
se enfoca en determinar si el modelo general de regresión se comporta
como el descrito a continuación: \begin{equation*}
Y_t= β_1  + β_2 t + β_3 Y_(t-1)  + u_t
\end{equation*} Donde \(u_t\) es un término de error de ruido blanco (se
distribuye normalmente con media = 0 y varianza = 1), \(t\) es el tiempo
medido cronológicamente, \(Y_t\) la realización de la serie, \(β_1\)
indica el valor de deriva, \(β_2\) la tendencia temporal y \(β_3\) la
raíz del proceso.

\textbf{2. Raíz unitaria con deriva y sin tendencia} Del modelo general
de regresión se puede presentar una forma especial en la cual \(β_1≠0\),
\(β_2=0\), \(β_3=1\), y se conoce como Caminata Aleatoria con Deriva
(CAD), ésta se caracteriza por ser no estacionaria con tendencia
determinista: \begin{equation*}
Y_t= β_1  +  β_3 Y_(t-1)  + u_t
\end{equation*}

\textbf{3. Raíz unitaria sin deriva y sin tendencia} Del modelo general
de regresión se puede presentar una forma especial en la cual \(β_1=0\),
\(β_2=0\), \(β_3=1\), y es conocido como Caminata Aleatoria Pura (CAP),
el cual genera un proceso no estacionario:

\begin{equation*}
Y_t= Y_(t-1)  + u_t
\end{equation*}

    \hypertarget{prueba-de-rachas-prueba-de-geary}{%
\subsubsection{2.1.2 Prueba de rachas (Prueba de
Geary)}\label{prueba-de-rachas-prueba-de-geary}}

    El cuarto test aplicado es una prueba no paramétrica que busca
determinar si las realizaciones en una serie siguen un comportamiento
aleatorio (Geary, 1935; Gujarati \& Porter, 2010), para ello se estima
un valor crítico Z teniendo en cuenta la siguiente ecuación:
\begin{equation*}
Z= R-X⁄σ^2 
\end{equation*} Donde \(Z\) es una variable que se distribuye de manera
Normal, \(R\) es el número total de rachas,
\(X= ((2*n^1*n^2 ))⁄((n)+1)\) (Media de la serie), \(n^1\) representa el
número total de rachas positivas, \(n^2\) el número total de rachas
negativas, \(n= (n^1+n^2 )\), \$ σ\^{}2=
((2\emph{n\textsuperscript{1\emph{n\textsuperscript{2\emph{(2}n}1}n}2-n)))⁄(((n)\^{}2}(n-1)))\$
(Varianza de la serie).

    \hypertarget{anuxe1lisis-de-auto-correlaciones-prueba-d-de-dubrin-watson}{%
\subsubsection{2.1.3 Análisis de Auto correlaciones (Prueba d, de
Dubrin-Watson)}\label{anuxe1lisis-de-auto-correlaciones-prueba-d-de-dubrin-watson}}

    La quinta prueba es el análisis de auto correlaciones de Durbin-Watson
(Savin \& White, 1977), ésta se utiliza para detectar la presencia de
auto correlación en los residuos (errores de predicción) de un análisis
de la regresión. El estadístico es calculado a continuación:

\begin{equation*}
d=\frac{\sum_{t=2}^T (e_t-e_(t-1))^2}{\sum_{t=1}^T(e_t)^2}
\end{equation*}

Donde \(d≈(2- β_3)\), i8ndicando que \(β_3\) representa el grado de
correlación de la muestra de los residuos. Un valor ``d = 2'' indica que
no hay auto correlación.

    \hypertarget{estimaciuxf3n-del-coeficiente-de-hurst}{%
\subsubsection{2.1.4 Estimación del coeficiente de
Hurst}\label{estimaciuxf3n-del-coeficiente-de-hurst}}

    La sexta prueba es la estimación del coeficiente de Hurst, éste es una
medida que indica el grado de independencia de las series de tiempo.
Cuando el valor del coeficiente se encuentre en \(H=0.5\), se considera
que la serie tiene un comportamiento de ruido blanco. Por otra parte, si
\((0.5≤H<1]\) se dice que la serie presenta persistencia o auto
correlaciones, de manera análoga, si \([0≤H≤0.5)\) la serie presenta un
comportamiento anti persistente o de correlación negativa. Durante el
presente trabajo es aplicada la variante propuesta por Andrew Lo, la
cual tiene en cuenta la memoria interna de la serie, donde
\$\{\hat\sigma\_x\}\^{}2 \$ y \$\{\hat y\_j\}\^{}2 \$ son los
estimadores de varianza y covarianza de la serie, \(a\) el tamaño de la
subserie, \(R(N)\) el rango de datos de la subserie:

\begin{equation*}
F'(N)= \frac{R(N)}{({\hat\sigma_x}^2 +2\sum_{j=1}^a(1-\frac{j}{a+1})*\hat{y_j})^2}
\end{equation*}

    \hypertarget{resultados-de-las-pruebas}{%
\subsection{2.2 Resultados de las
pruebas}\label{resultados-de-las-pruebas}}

    \begin{Verbatim}[commandchars=\\\{\}]
{\color{incolor}In [{\color{incolor}221}]:} \PY{c+c1}{\PYZsh{}Con el fin de evitar sobre utilizar bucles, para esta etapa se programará la estimación de las pruebas, y éstas se imprimiran posteriormente}
          
          \PY{c+c1}{\PYZsh{}Se construye una variable donde se almacenarán los resultados}
          COEF\PY{o}{\PYZlt{}\PYZhy{}}\PY{k+kp}{as.data.frame}\PY{p}{(}\PY{k+kp}{replicate}\PY{p}{(}\PY{k+kp}{length}\PY{p}{(}DATA\PY{p}{[}\PY{l+m}{1}\PY{p}{,}\PY{p}{]}\PY{p}{)}\PY{l+m}{\PYZhy{}1}\PY{p}{,} \PY{l+m}{0}\PY{o}{*}\PY{p}{(}\PY{l+m}{1}\PY{o}{:}\PY{l+m}{6}\PY{p}{)}\PY{p}{)}\PY{p}{)}
          
          \PY{c+c1}{\PYZsh{}Se programa una función que estime el coeficiente de Hurst}
          \PY{c+c1}{\PYZsh{}Inicio de la función}
          hurst \PY{o}{\PYZlt{}\PYZhy{}} \PY{k+kr}{function}\PY{p}{(}x\PY{p}{)}
          \PY{p}{\PYZob{}}
            
            N \PY{o}{\PYZlt{}\PYZhy{}} \PY{k+kp}{length}\PY{p}{(}x\PY{p}{)}
            s \PY{o}{\PYZlt{}\PYZhy{}} x\PY{p}{[}\PY{l+m}{2}\PY{o}{:}N\PY{p}{]}
            t \PY{o}{\PYZlt{}\PYZhy{}}\PY{l+m}{1}\PY{o}{:}N
            par1\PY{o}{\PYZlt{}\PYZhy{}}\PY{l+m}{3}
            RS\PY{o}{\PYZlt{}\PYZhy{}}\PY{l+m}{0}\PY{o}{*}\PY{p}{(}\PY{l+m}{1}\PY{o}{:}par1\PY{p}{)}
            tau\PY{o}{\PYZlt{}\PYZhy{}}RS
            \PY{k+kr}{for}\PY{p}{(}i \PY{k+kr}{in} \PY{l+m}{1}\PY{o}{:}par1\PY{p}{)}\PY{p}{\PYZob{}}
              m \PY{o}{\PYZlt{}\PYZhy{}} \PY{k+kp}{floor}\PY{p}{(}N\PY{o}{/}\PY{p}{(}i\PY{p}{)}\PY{p}{)}
              R\PY{o}{\PYZlt{}\PYZhy{}}\PY{l+m}{0}\PY{o}{*}\PY{p}{(}\PY{l+m}{1}\PY{o}{:}i\PY{p}{)}
              S\PY{o}{\PYZlt{}\PYZhy{}}\PY{l+m}{0}\PY{o}{*}\PY{p}{(}\PY{l+m}{1}\PY{o}{:}i\PY{p}{)}
              Var\PY{o}{\PYZlt{}\PYZhy{}}R
              \PY{k+kr}{for}\PY{p}{(}j \PY{k+kr}{in} \PY{l+m}{1}\PY{o}{:}i\PY{p}{)}\PY{p}{\PYZob{}}
                smin\PY{o}{\PYZlt{}\PYZhy{}}\PY{l+m}{1}\PY{o}{+}\PY{p}{(}j\PY{l+m}{\PYZhy{}1}\PY{p}{)}\PY{o}{*}m
                smax\PY{o}{\PYZlt{}\PYZhy{}}j\PY{o}{*}m\PY{l+m}{\PYZhy{}1}
                r \PY{o}{\PYZlt{}\PYZhy{}} s\PY{p}{[}smin\PY{o}{:}smax\PY{p}{]}
                M\PY{o}{\PYZlt{}\PYZhy{}}\PY{k+kp}{mean}\PY{p}{(}r\PY{p}{)}
                x2\PY{o}{\PYZlt{}\PYZhy{}}r\PY{o}{\PYZhy{}}M
                V \PY{o}{\PYZlt{}\PYZhy{}} \PY{k+kp}{cumsum}\PY{p}{(}x2\PY{p}{)}
                R\PY{p}{[}j\PY{p}{]}\PY{o}{\PYZlt{}\PYZhy{}}\PY{k+kp}{max}\PY{p}{(}V\PY{p}{)}\PY{o}{\PYZhy{}}\PY{k+kp}{min}\PY{p}{(}V\PY{p}{)}
                Var\PY{p}{[}j\PY{p}{]}\PY{o}{\PYZlt{}\PYZhy{}}var\PY{p}{(}r\PY{p}{)}
                wj\PY{o}{\PYZlt{}\PYZhy{}}\PY{p}{(}\PY{l+m}{1}\PY{o}{\PYZhy{}}\PY{p}{(}j\PY{o}{/}\PY{p}{(}i\PY{l+m}{+1}\PY{p}{)}\PY{p}{)}\PY{p}{)}
                x22\PY{o}{\PYZlt{}\PYZhy{}}\PY{l+m}{0}
                \PY{k+kr}{for}\PY{p}{(}z \PY{k+kr}{in} \PY{l+m}{1}\PY{o}{:}\PY{p}{(}\PY{k+kp}{length}\PY{p}{(}x2\PY{p}{)}\PY{l+m}{\PYZhy{}1}\PY{p}{)}\PY{p}{)}\PY{p}{\PYZob{}}
                  x22\PY{p}{[}z\PY{p}{]}\PY{o}{\PYZlt{}\PYZhy{}}x2\PY{p}{[}z\PY{p}{]}\PY{o}{*}x2\PY{p}{[}z\PY{l+m}{+1}\PY{p}{]}
                \PY{p}{\PYZcb{}}
                cova\PY{o}{\PYZlt{}\PYZhy{}}\PY{k+kp}{sum}\PY{p}{(}x22\PY{p}{)}
                S\PY{p}{[}j\PY{p}{]} \PY{o}{\PYZlt{}\PYZhy{}} \PY{p}{(}Var\PY{p}{[}j\PY{p}{]} \PY{o}{+}  \PY{l+m}{2}\PY{o}{*}wj\PY{o}{*}cova\PY{o}{/}m\PY{p}{)}\PY{o}{\PYZca{}}\PY{p}{(}\PY{l+m}{0.5}\PY{p}{)}
              \PY{p}{\PYZcb{}}
              tau\PY{p}{[}i\PY{p}{]}\PY{o}{\PYZlt{}\PYZhy{}}m
              RS\PY{p}{[}i\PY{p}{]}\PY{o}{\PYZlt{}\PYZhy{}}\PY{k+kp}{mean}\PY{p}{(}R\PY{o}{/}S\PY{p}{)} 
            \PY{p}{\PYZcb{}}
            XX\PY{o}{\PYZlt{}\PYZhy{}} \PY{k+kt}{data.frame}\PY{p}{(}\PY{k+kp}{log10}\PY{p}{(}tau\PY{p}{)}\PY{p}{,}\PY{k+kp}{log10}\PY{p}{(}RS\PY{p}{)}\PY{p}{)}
            x \PY{o}{\PYZlt{}\PYZhy{}} XX\PY{o}{\PYZdl{}}log10.tau.
            y \PY{o}{\PYZlt{}\PYZhy{}} XX\PY{o}{\PYZdl{}}log10.RS.
            n \PY{o}{\PYZlt{}\PYZhy{}} \PY{k+kp}{nrow}\PY{p}{(}XX\PY{p}{)}
            xy \PY{o}{\PYZlt{}\PYZhy{}} x\PY{o}{*}y
            m2 \PY{o}{\PYZlt{}\PYZhy{}} \PY{p}{(}n\PY{o}{*}\PY{k+kp}{sum}\PY{p}{(}xy\PY{p}{)}\PY{o}{\PYZhy{}}\PY{k+kp}{sum}\PY{p}{(}x\PY{p}{)}\PY{o}{*}\PY{k+kp}{sum}\PY{p}{(}y\PY{p}{)}\PY{p}{)} \PY{o}{/} \PY{p}{(}n\PY{o}{*}\PY{k+kp}{sum}\PY{p}{(}x\PY{o}{\PYZca{}}\PY{l+m}{2}\PY{p}{)}\PY{o}{\PYZhy{}}\PY{k+kp}{sum}\PY{p}{(}x\PY{p}{)}\PY{o}{\PYZca{}}\PY{l+m}{2}\PY{p}{)}
            m2
            
            \PY{k+kr}{return}\PY{p}{(}m2\PY{p}{)}
          \PY{p}{\PYZcb{}}
          \PY{c+c1}{\PYZsh{}Fin de la función}
          
          \PY{c+c1}{\PYZsh{}Se inicia un bucle que recorra el listado de productos}
          \PY{k+kr}{for}\PY{p}{(}i \PY{k+kr}{in} \PY{l+m}{1}\PY{o}{:}\PY{p}{(}\PY{k+kp}{length}\PY{p}{(}DATA\PY{p}{[}\PY{l+m}{1}\PY{p}{,}\PY{p}{]}\PY{p}{)}\PY{l+m}{\PYZhy{}1}\PY{p}{)}\PY{p}{)}
          \PY{p}{\PYZob{}}
              \PY{c+c1}{\PYZsh{}Prueba de raíz unitaria}
              \PY{c+c1}{\PYZsh{} Si el p\PYZhy{}valor es menor que 0.05 no podemos rechazar:}
              
              \PY{c+c1}{\PYZsh{}se transforman los datos}
              DATA\PY{p}{[}\PY{p}{,}i\PY{l+m}{+1}\PY{p}{]}\PY{o}{\PYZlt{}\PYZhy{}}\PY{k+kp}{as.data.frame.vector}\PY{p}{(}DATA\PY{p}{[}\PY{p}{,}i\PY{l+m}{+1}\PY{p}{]}\PY{p}{)}
              
              \PY{c+c1}{\PYZsh{}Se aplican todas las pruebas, bajo el supuesto de la existencia de una tendencia, para evaluar las 3 hipótesis}
              \PY{c+c1}{\PYZsh{}los datos son almacenados en la variable COEF para las tres primeras filas}
          
              df\PY{o}{\PYZlt{}\PYZhy{}}ur.df\PY{p}{(}DATA\PY{p}{[}\PY{p}{,}i\PY{l+m}{+1}\PY{p}{]}\PY{p}{,}type\PY{o}{=}\PY{l+s}{\PYZdq{}}\PY{l+s}{trend\PYZdq{}}\PY{p}{,}lags\PY{o}{=}\PY{l+m}{0}\PY{p}{,} selectlags \PY{o}{=} \PY{l+s}{\PYZdq{}}\PY{l+s}{AIC\PYZdq{}}\PY{p}{)}\PY{c+c1}{\PYZsh{}tendencia, }
              COEF\PY{p}{[}\PY{l+m}{1}\PY{p}{,}i\PY{p}{]}\PY{o}{\PYZlt{}\PYZhy{}}\PY{k+kp}{summary}\PY{p}{(}df\PY{p}{)}\PY{o}{@}testreg\PY{o}{\PYZdl{}}coefficients\PY{p}{[}\PY{l+m}{10}\PY{p}{]} \PY{c+c1}{\PYZsh{}Hipótesis nula, Hay raiz unitaria}
              COEF\PY{p}{[}\PY{l+m}{2}\PY{p}{,}i\PY{p}{]}\PY{o}{\PYZlt{}\PYZhy{}}\PY{k+kp}{summary}\PY{p}{(}df\PY{p}{)}\PY{o}{@}testreg\PY{o}{\PYZdl{}}coefficients\PY{p}{[}\PY{l+m}{11}\PY{p}{]} \PY{c+c1}{\PYZsh{}Hipótesis nula, Hay raiz unitaria sin tendencia}
              COEF\PY{p}{[}\PY{l+m}{3}\PY{p}{,}i\PY{p}{]}\PY{o}{\PYZlt{}\PYZhy{}}\PY{k+kp}{summary}\PY{p}{(}df\PY{p}{)}\PY{o}{@}testreg\PY{o}{\PYZdl{}}coefficients\PY{p}{[}\PY{l+m}{12}\PY{p}{]} \PY{c+c1}{\PYZsh{}Hipótesis nula, Hay raiz unitaria sin tendencia y sin deriva}
                
              \PY{c+c1}{\PYZsh{}Prueba de rachas}
              \PY{c+c1}{\PYZsh{}si P valor \PYZgt{}5\PYZpc{}, no hay evidencias para considerar los datos no aleatorios.}
              xr\PY{o}{\PYZlt{}\PYZhy{}}DATA\PY{p}{[}\PY{l+m}{2}\PY{o}{:}\PY{k+kp}{length}\PY{p}{(}DATA\PY{p}{[}\PY{p}{,}i\PY{l+m}{+1}\PY{p}{]}\PY{p}{)}\PY{p}{,}i\PY{l+m}{+1}\PY{p}{]}\PY{o}{/}DATA\PY{p}{[}\PY{l+m}{1}\PY{o}{:}\PY{p}{(}\PY{k+kp}{length}\PY{p}{(}DATA\PY{p}{[}\PY{p}{,}i\PY{l+m}{+1}\PY{p}{]}\PY{p}{)}\PY{l+m}{\PYZhy{}1}\PY{p}{)}\PY{p}{,}i\PY{l+m}{+1}\PY{p}{]}
              a\PY{o}{\PYZlt{}\PYZhy{}} randtests\PY{o}{::}runs.test\PY{p}{(}xr\PY{p}{)}
              
              COEF\PY{p}{[}\PY{l+m}{4}\PY{p}{,}i\PY{p}{]}\PY{o}{\PYZlt{}\PYZhy{}}a\PY{o}{\PYZdl{}}p.value
              \PY{c+c1}{\PYZsh{}Autocorrelaciones}
              \PY{c+c1}{\PYZsh{}si el pvalor es mayor a 0.05, se rechaza la hipótesis nula (Autocorrelación=0)}
              x1\PY{o}{\PYZlt{}\PYZhy{}}\PY{l+m}{1}\PY{o}{:}\PY{k+kp}{length}\PY{p}{(}DATA\PY{p}{[}\PY{p}{,}\PY{l+m}{1}\PY{p}{]}\PY{p}{)}
              aut\PY{o}{\PYZlt{}\PYZhy{}}dwtest\PY{p}{(}DATA\PY{p}{[}\PY{p}{,}i\PY{l+m}{+1}\PY{p}{]} \PY{o}{\PYZti{}} x1\PY{p}{)} 
              COEF\PY{p}{[}\PY{l+m}{5}\PY{p}{,}i\PY{p}{]}\PY{o}{\PYZlt{}\PYZhy{}}aut\PY{o}{\PYZdl{}}p.value
            
          
              \PY{c+c1}{\PYZsh{}Coeficiente de Hurst}
              
              \PY{c+c1}{\PYZsh{}Se transforman los datos para determinar su rendimiento}
              xr\PY{o}{\PYZlt{}\PYZhy{}}DATA\PY{p}{[}\PY{l+m}{2}\PY{o}{:}\PY{k+kp}{length}\PY{p}{(}DATA\PY{p}{[}\PY{p}{,}i\PY{l+m}{+1}\PY{p}{]}\PY{p}{)}\PY{p}{,}i\PY{l+m}{+1}\PY{p}{]}\PY{o}{/}DATA\PY{p}{[}\PY{l+m}{1}\PY{o}{:}\PY{p}{(}\PY{k+kp}{length}\PY{p}{(}DATA\PY{p}{[}\PY{p}{,}i\PY{l+m}{+1}\PY{p}{]}\PY{p}{)}\PY{l+m}{\PYZhy{}1}\PY{p}{)}\PY{p}{,}i\PY{l+m}{+1}\PY{p}{]}
              \PY{c+c1}{\PYZsh{}Se aplica la función para estimar el coeficiente de Hurst y se almacena en la sexta posición}
              COEF\PY{p}{[}\PY{l+m}{6}\PY{p}{,}i\PY{p}{]}\PY{o}{\PYZlt{}\PYZhy{}}hurst\PY{p}{(}xr\PY{p}{)}
            
            
          \PY{p}{\PYZcb{}}
          \PY{c+c1}{\PYZsh{}Se cambia el formato de los valores}
          COEF\PY{o}{\PYZlt{}\PYZhy{}}\PY{k+kp}{format}\PY{p}{(}COEF\PY{p}{,} scientific\PY{o}{=}\PY{k+kc}{FALSE}\PY{p}{)}
          \PY{c+c1}{\PYZsh{}Se guardan los valores d emanera local}
          write.csv\PY{p}{(}COEF\PY{p}{,} file\PY{o}{=}\PY{l+s}{\PYZdq{}}\PY{l+s}{COEFICIENTES\PYZus{}FINAL.csv\PYZdq{}}\PY{p}{)}  
          
          \PY{c+c1}{\PYZsh{}Se transforma el formato de los valores para poder hacer cálculos}
          COEF\PY{o}{\PYZlt{}\PYZhy{}}\PY{k+kp}{data.matrix}\PY{p}{(}COEF\PY{p}{,} rownames.force \PY{o}{=} \PY{k+kc}{NA}\PY{p}{)}
          \PY{c+c1}{\PYZsh{}Para mejorar la visualziación, se truncan los valores}
          COEF\PY{o}{\PYZlt{}\PYZhy{}}\PY{k+kp}{round}\PY{p}{(}COEF\PY{p}{,}\PY{l+m}{3}\PY{p}{)}
          
          \PY{c+c1}{\PYZsh{}Con el fin de presentar cada prueba por separado, se transforma y transpone la matriz de coeficientes}
          n\PY{o}{\PYZlt{}\PYZhy{}}\PY{k+kp}{colnames}\PY{p}{(}DATA\PY{p}{[}\PY{p}{,}\PY{l+m}{\PYZhy{}1}\PY{p}{]}\PY{p}{)}
          
          DATA2 \PY{o}{\PYZlt{}\PYZhy{}} \PY{k+kp}{as.data.frame}\PY{p}{(}\PY{k+kp}{t}\PY{p}{(}COEF\PY{p}{[}\PY{p}{,}\PY{p}{]}\PY{p}{)}\PY{p}{)}
          \PY{k+kp}{names}\PY{p}{(}DATA2\PY{p}{)}\PY{p}{[}\PY{l+m}{1}\PY{p}{]}\PY{o}{\PYZlt{}\PYZhy{}}\PY{k+kp}{paste}\PY{p}{(}\PY{l+s}{\PYZdq{}}\PY{l+s}{Raiz Unitaria\PYZdq{}}\PY{p}{)}
          \PY{k+kp}{names}\PY{p}{(}DATA2\PY{p}{)}\PY{p}{[}\PY{l+m}{2}\PY{p}{]}\PY{o}{\PYZlt{}\PYZhy{}}\PY{k+kp}{paste}\PY{p}{(}\PY{l+s}{\PYZdq{}}\PY{l+s}{Raiz unitaria sin tendencia\PYZdq{}}\PY{p}{)}
          \PY{k+kp}{names}\PY{p}{(}DATA2\PY{p}{)}\PY{p}{[}\PY{l+m}{3}\PY{p}{]}\PY{o}{\PYZlt{}\PYZhy{}}\PY{k+kp}{paste}\PY{p}{(}\PY{l+s}{\PYZdq{}}\PY{l+s}{Raiz unitaria sin tendencia y sin deriva\PYZdq{}}\PY{p}{)}
          \PY{k+kp}{names}\PY{p}{(}DATA2\PY{p}{)}\PY{p}{[}\PY{l+m}{4}\PY{p}{]}\PY{o}{\PYZlt{}\PYZhy{}}\PY{k+kp}{paste}\PY{p}{(}\PY{l+s}{\PYZdq{}}\PY{l+s}{Raiz Prueba de rachas\PYZdq{}}\PY{p}{)}
          \PY{k+kp}{names}\PY{p}{(}DATA2\PY{p}{)}\PY{p}{[}\PY{l+m}{5}\PY{p}{]}\PY{o}{\PYZlt{}\PYZhy{}}\PY{k+kp}{paste}\PY{p}{(}\PY{l+s}{\PYZdq{}}\PY{l+s}{Raiz Autocorrelaciones\PYZdq{}}\PY{p}{)}
          \PY{k+kp}{names}\PY{p}{(}DATA2\PY{p}{)}\PY{p}{[}\PY{l+m}{6}\PY{p}{]}\PY{o}{\PYZlt{}\PYZhy{}}\PY{k+kp}{paste}\PY{p}{(}\PY{l+s}{\PYZdq{}}\PY{l+s}{Coeficiente de Hurst\PYZdq{}}\PY{p}{)}
          DATA2\PY{o}{\PYZdl{}}NOMBRE\PY{o}{\PYZlt{}\PYZhy{}}n
          \PY{k+kp}{row.names}\PY{p}{(}DATA2\PY{p}{)} \PY{o}{\PYZlt{}\PYZhy{}} n
          DATA2
\end{Verbatim}


    \begin{tabular}{r|lllllll}
  & Raiz Unitaria & Raiz unitaria sin tendencia & Raiz unitaria sin tendencia y sin deriva & Raiz Prueba de rachas & Raiz Autocorrelaciones & Coeficiente de Hurst & NOMBRE\\
\hline
	Ahuyama & 0.010            & 0.003            & 0.420            & 0.083            & 0                & 0.323            & Ahuyama         \\
	Ajo & 0.190            & 0.019            & 0.045            & 0.001            & 0                & 0.159            & Ajo             \\
	Arverja & 0.016            & 0.001            & 0.227            & 0.384            & 0                & 0.268            & Arverja         \\
	Cebolla.Cabezona & 0.054            & 0.003            & 0.283            & 0.013            & 0                & 0.297            & Cebolla.Cabezona\\
	Cebolla.junca & 0.013            & 0.000            & 0.015            & 0.000            & 0                & 0.279            & Cebolla.junca   \\
	Cilantro & 0.000            & 0.000            & 0.170            & 0.618            & 0                & 0.275            & Cilantro        \\
	Cebada & 0.000            & 0.000            & 0.000            & 0.000            & 0                & 0.378            & Cebada          \\
	Maiz & 0.000            & 0.000            & 0.000            & 0.000            & 0                & 0.083            & Maiz            \\
	Frijol & 0.000            & 0.000            & 0.299            & 0.709            & 0                & 0.333            & Frijol          \\
	Habichuela & 0.000            & 0.000            & 0.480            & 0.709            & 0                & 0.155            & Habichuela      \\
	Lechuga & 0.000            & 0.000            & 0.477            & 0.080            & 0                & 0.296            & Lechuga         \\
	Perejil & 0.001            & 0.000            & 0.131            & 0.270            & 0                & 0.236            & Perejil         \\
	Pimenton & 0.001            & 0.000            & 0.072            & 0.106            & 0                & 0.318            & Pimenton        \\
	Repollo.blanco & 0.001            & 0.000            & 0.985            & 0.901            & 0                & 0.119            & Repollo.blanco  \\
	Tomate & 0.001            & 0.000            & 0.374            & 0.901            & 0                & 0.177            & Tomate          \\
	Banano & 0.000            & 0.000            & 0.003            & 0.544            & 0                & 0.351            & Banano          \\
	Lulo & 0.002            & 0.002            & 0.719            & 0.901            & 0                & 0.468            & Lulo            \\
	Mora & 0.000            & 0.000            & 0.640            & 0.456            & 0                & 0.186            & Mora            \\
	Papaya & 0.004            & 0.003            & 0.639            & 0.000            & 0                & 0.391            & Papaya          \\
	Patilla & 0.000            & 0.000            & 0.039            & 0.018            & 0                & 0.187            & Patilla         \\
	Pinna & 0.000            & 0.000            & 0.154            & 0.106            & 0                & 0.165            & Pinna           \\
	Arracacha & 0.262            & 0.028            & 0.499            & 0.000            & 0                & 0.510            & Arracacha       \\
	Papa & 0.050            & 0.058            & 0.840            & 0.099            & 0                & 0.154            & Papa            \\
	Plátano & 0.000            & 0.000            & 0.616            & 0.047            & 0                & 0.110            & Plátano         \\
	Yuca & 0.015            & 0.001            & 0.143            & 0.001            & 0                & 0.611            & Yuca            \\
\end{tabular}


    
    \hypertarget{prueba-de-rauxedz-unitaria}{%
\subsubsection{2.2.1 Prueba de raíz
unitaria}\label{prueba-de-rauxedz-unitaria}}

    \begin{Verbatim}[commandchars=\\\{\}]
{\color{incolor}In [{\color{incolor}242}]:} \PY{k+kp}{print}\PY{p}{(}\PY{l+s}{\PYZdq{}}\PY{l+s}{Estos son los productos que presentan un comportamiento de raiz unitaria\PYZdq{}}\PY{p}{)}
          \PY{k+kp}{print}\PY{p}{(}\PY{l+s}{\PYZdq{}}\PY{l+s}{Total de productos:\PYZdq{}}\PY{p}{)}
          \PY{k+kp}{length}\PY{p}{(}DATA2\PY{p}{[}\PY{k+kp}{which}\PY{p}{(}DATA2\PY{p}{[}\PY{p}{,}\PY{l+m}{1}\PY{p}{]}\PY{o}{\PYZlt{}=}\PY{l+m}{0.05}\PY{p}{)}\PY{p}{,}\PY{l+m}{7}\PY{p}{]}\PY{p}{)}
          \PY{k+kp}{print}\PY{p}{(}\PY{l+s}{\PYZdq{}}\PY{l+s}{Nombre de los productos:\PYZdq{}}\PY{p}{)}
          DATA2\PY{p}{[}\PY{k+kp}{which}\PY{p}{(}DATA2\PY{p}{[}\PY{p}{,}\PY{l+m}{1}\PY{p}{]}\PY{o}{\PYZlt{}=}\PY{l+m}{0.05}\PY{p}{)}\PY{p}{,}\PY{l+m}{7}\PY{p}{]}
          \PY{k+kp}{print}\PY{p}{(}\PY{l+s}{\PYZdq{}}\PY{l+s}{Estos son los productos que presentan un comportamiento de raiz unitaria sin tendencia\PYZdq{}}\PY{p}{)}
          \PY{k+kp}{print}\PY{p}{(}\PY{l+s}{\PYZdq{}}\PY{l+s}{Total de productos:\PYZdq{}}\PY{p}{)}
          \PY{k+kp}{length}\PY{p}{(}DATA2\PY{p}{[}\PY{k+kp}{which}\PY{p}{(}DATA2\PY{p}{[}\PY{p}{,}\PY{l+m}{2}\PY{p}{]}\PY{o}{\PYZlt{}=}\PY{l+m}{0.05}\PY{p}{)}\PY{p}{,}\PY{l+m}{7}\PY{p}{]}\PY{p}{)}
          \PY{k+kp}{print}\PY{p}{(}\PY{l+s}{\PYZdq{}}\PY{l+s}{Nombre de los productos:\PYZdq{}}\PY{p}{)}
          DATA2\PY{p}{[}\PY{k+kp}{which}\PY{p}{(}DATA2\PY{p}{[}\PY{p}{,}\PY{l+m}{2}\PY{p}{]}\PY{o}{\PYZlt{}=}\PY{l+m}{0.05}\PY{p}{)}\PY{p}{,}\PY{l+m}{7}\PY{p}{]}
          \PY{k+kp}{print}\PY{p}{(}\PY{l+s}{\PYZdq{}}\PY{l+s}{Estos son los productos que presentan un comportamiento de raiz unitaria sin tendencia y sin deriva\PYZdq{}}\PY{p}{)}
          \PY{k+kp}{print}\PY{p}{(}\PY{l+s}{\PYZdq{}}\PY{l+s}{Total de productos:\PYZdq{}}\PY{p}{)}
          \PY{k+kp}{length}\PY{p}{(}DATA2\PY{p}{[}\PY{k+kp}{which}\PY{p}{(}DATA2\PY{p}{[}\PY{p}{,}\PY{l+m}{3}\PY{p}{]}\PY{o}{\PYZlt{}=}\PY{l+m}{0.05}\PY{p}{)}\PY{p}{,}\PY{l+m}{7}\PY{p}{]}\PY{p}{)}
          \PY{k+kp}{print}\PY{p}{(}\PY{l+s}{\PYZdq{}}\PY{l+s}{Nombre de los productos:\PYZdq{}}\PY{p}{)}
          DATA2\PY{p}{[}\PY{k+kp}{which}\PY{p}{(}DATA2\PY{p}{[}\PY{p}{,}\PY{l+m}{3}\PY{p}{]}\PY{o}{\PYZlt{}=}\PY{l+m}{0.05}\PY{p}{)}\PY{p}{,}\PY{l+m}{7}\PY{p}{]}
\end{Verbatim}


    \begin{Verbatim}[commandchars=\\\{\}]
[1] "Estos son los productos que presentan un comportamiento de raiz unitaria"
[1] "Total de productos:"

    \end{Verbatim}

    22

    
    \begin{Verbatim}[commandchars=\\\{\}]
[1] "Nombre de los productos:"

    \end{Verbatim}

    \begin{enumerate*}
\item 'Ahuyama'
\item 'Arverja'
\item 'Cebolla.junca'
\item 'Cilantro'
\item 'Cebada'
\item 'Maiz'
\item 'Frijol'
\item 'Habichuela'
\item 'Lechuga'
\item 'Perejil'
\item 'Pimenton'
\item 'Repollo.blanco'
\item 'Tomate'
\item 'Banano'
\item 'Lulo'
\item 'Mora'
\item 'Papaya'
\item 'Patilla'
\item 'Pinna'
\item 'Papa'
\item 'Plátano'
\item 'Yuca'
\end{enumerate*}


    
    \begin{Verbatim}[commandchars=\\\{\}]
[1] "Estos son los productos que presentan un comportamiento de raiz unitaria sin tendencia"
[1] "Total de productos:"

    \end{Verbatim}

    24

    
    \begin{Verbatim}[commandchars=\\\{\}]
[1] "Nombre de los productos:"

    \end{Verbatim}

    \begin{enumerate*}
\item 'Ahuyama'
\item 'Ajo'
\item 'Arverja'
\item 'Cebolla.Cabezona'
\item 'Cebolla.junca'
\item 'Cilantro'
\item 'Cebada'
\item 'Maiz'
\item 'Frijol'
\item 'Habichuela'
\item 'Lechuga'
\item 'Perejil'
\item 'Pimenton'
\item 'Repollo.blanco'
\item 'Tomate'
\item 'Banano'
\item 'Lulo'
\item 'Mora'
\item 'Papaya'
\item 'Patilla'
\item 'Pinna'
\item 'Arracacha'
\item 'Plátano'
\item 'Yuca'
\end{enumerate*}


    
    \begin{Verbatim}[commandchars=\\\{\}]
[1] "Estos son los productos que presentan un comportamiento de raiz unitaria sin tendencia y sin deriva"
[1] "Total de productos:"

    \end{Verbatim}

    6

    
    \begin{Verbatim}[commandchars=\\\{\}]
[1] "Nombre de los productos:"

    \end{Verbatim}

    \begin{enumerate*}
\item 'Ajo'
\item 'Cebolla.junca'
\item 'Cebada'
\item 'Maiz'
\item 'Banano'
\item 'Patilla'
\end{enumerate*}


    
    \hypertarget{prueba-de-rachas}{%
\subsubsection{2.2.2 Prueba de rachas}\label{prueba-de-rachas}}

    \begin{Verbatim}[commandchars=\\\{\}]
{\color{incolor}In [{\color{incolor}243}]:} \PY{k+kp}{print}\PY{p}{(}\PY{l+s}{\PYZdq{}}\PY{l+s}{Estos son los productos que presentan un comportamiento aleatorio\PYZdq{}}\PY{p}{)}
          \PY{k+kp}{print}\PY{p}{(}\PY{l+s}{\PYZdq{}}\PY{l+s}{Total de productos:\PYZdq{}}\PY{p}{)}
          \PY{k+kp}{length}\PY{p}{(}DATA2\PY{p}{[}\PY{k+kp}{which}\PY{p}{(}DATA2\PY{p}{[}\PY{p}{,}\PY{l+m}{4}\PY{p}{]}\PY{o}{\PYZlt{}=}\PY{l+m}{0.05}\PY{p}{)}\PY{p}{,}\PY{l+m}{7}\PY{p}{]}\PY{p}{)}
          \PY{k+kp}{print}\PY{p}{(}\PY{l+s}{\PYZdq{}}\PY{l+s}{Nombre de los productos:\PYZdq{}}\PY{p}{)}
          DATA2\PY{p}{[}\PY{k+kp}{which}\PY{p}{(}DATA2\PY{p}{[}\PY{p}{,}\PY{l+m}{4}\PY{p}{]}\PY{o}{\PYZlt{}=}\PY{l+m}{0.05}\PY{p}{)}\PY{p}{,}\PY{l+m}{7}\PY{p}{]}
\end{Verbatim}


    \begin{Verbatim}[commandchars=\\\{\}]
[1] "Estos son los productos que presentan un comportamiento aleatorio"
[1] "Total de productos:"

    \end{Verbatim}

    10

    
    \begin{Verbatim}[commandchars=\\\{\}]
[1] "Nombre de los productos:"

    \end{Verbatim}

    \begin{enumerate*}
\item 'Ajo'
\item 'Cebolla.Cabezona'
\item 'Cebolla.junca'
\item 'Cebada'
\item 'Maiz'
\item 'Papaya'
\item 'Patilla'
\item 'Arracacha'
\item 'Plátano'
\item 'Yuca'
\end{enumerate*}


    
    \hypertarget{anuxe1lisis-de-autocorrelaciones}{%
\subsubsection{2.2.3 Análisis de
autocorrelaciones}\label{anuxe1lisis-de-autocorrelaciones}}

    \begin{Verbatim}[commandchars=\\\{\}]
{\color{incolor}In [{\color{incolor}244}]:} \PY{k+kp}{print}\PY{p}{(}\PY{l+s}{\PYZdq{}}\PY{l+s}{Estos son los productos que presentan una autocorrelación\PYZdq{}}\PY{p}{)}
          \PY{k+kp}{print}\PY{p}{(}\PY{l+s}{\PYZdq{}}\PY{l+s}{Total de productos:\PYZdq{}}\PY{p}{)}
          \PY{k+kp}{length}\PY{p}{(}DATA2\PY{p}{[}\PY{k+kp}{which}\PY{p}{(}DATA2\PY{p}{[}\PY{p}{,}\PY{l+m}{5}\PY{p}{]}\PY{o}{\PYZlt{}=}\PY{l+m}{0.05}\PY{p}{)}\PY{p}{,}\PY{l+m}{7}\PY{p}{]}\PY{p}{)}
          \PY{k+kp}{print}\PY{p}{(}\PY{l+s}{\PYZdq{}}\PY{l+s}{Nombre de los productos:\PYZdq{}}\PY{p}{)}
          DATA2\PY{p}{[}\PY{k+kp}{which}\PY{p}{(}DATA2\PY{p}{[}\PY{p}{,}\PY{l+m}{5}\PY{p}{]}\PY{o}{\PYZlt{}=}\PY{l+m}{0.05}\PY{p}{)}\PY{p}{,}\PY{l+m}{7}\PY{p}{]}
\end{Verbatim}


    \begin{Verbatim}[commandchars=\\\{\}]
[1] "Estos son los productos que presentan una autocorrelación"
[1] "Total de productos:"

    \end{Verbatim}

    25

    
    \begin{Verbatim}[commandchars=\\\{\}]
[1] "Nombre de los productos:"

    \end{Verbatim}

    \begin{enumerate*}
\item 'Ahuyama'
\item 'Ajo'
\item 'Arverja'
\item 'Cebolla.Cabezona'
\item 'Cebolla.junca'
\item 'Cilantro'
\item 'Cebada'
\item 'Maiz'
\item 'Frijol'
\item 'Habichuela'
\item 'Lechuga'
\item 'Perejil'
\item 'Pimenton'
\item 'Repollo.blanco'
\item 'Tomate'
\item 'Banano'
\item 'Lulo'
\item 'Mora'
\item 'Papaya'
\item 'Patilla'
\item 'Pinna'
\item 'Arracacha'
\item 'Papa'
\item 'Plátano'
\item 'Yuca'
\end{enumerate*}


    
    \hypertarget{estimaciuxf3n-del-coeficiente-de-hurst}{%
\subsubsection{2.2.4 Estimación del coeficiente de
Hurst}\label{estimaciuxf3n-del-coeficiente-de-hurst}}

    \begin{Verbatim}[commandchars=\\\{\}]
{\color{incolor}In [{\color{incolor}247}]:} \PY{k+kp}{print}\PY{p}{(}\PY{l+s}{\PYZdq{}}\PY{l+s}{Estos son los productos que presentan un comportamiento de ruido blanco (H aproximadamente 0.5)\PYZdq{}}\PY{p}{)}
          \PY{k+kp}{print}\PY{p}{(}\PY{l+s}{\PYZdq{}}\PY{l+s}{Total de productos:\PYZdq{}}\PY{p}{)}
          \PY{k+kp}{length}\PY{p}{(}DATA2\PY{p}{[}\PY{k+kp}{which}\PY{p}{(}DATA2\PY{p}{[}\PY{p}{,}\PY{l+m}{6}\PY{p}{]}\PY{o}{\PYZgt{}=}\PY{l+m}{0.4804} \PY{o}{|} DATA2\PY{p}{[}\PY{p}{,}\PY{l+m}{6}\PY{p}{]}\PY{o}{\PYZgt{}=}\PY{l+m}{0.5245}\PY{p}{)}\PY{p}{,}\PY{l+m}{7}\PY{p}{]}\PY{p}{)}
          \PY{k+kp}{print}\PY{p}{(}\PY{l+s}{\PYZdq{}}\PY{l+s}{Nombre de los productos:\PYZdq{}}\PY{p}{)}
          DATA2\PY{p}{[}\PY{k+kp}{which}\PY{p}{(}DATA2\PY{p}{[}\PY{p}{,}\PY{l+m}{6}\PY{p}{]}\PY{o}{\PYZgt{}=}\PY{l+m}{0.4804} \PY{o}{|} DATA2\PY{p}{[}\PY{p}{,}\PY{l+m}{6}\PY{p}{]}\PY{o}{\PYZgt{}=}\PY{l+m}{0.5245}\PY{p}{)}\PY{p}{,}\PY{l+m}{7}\PY{p}{]}
\end{Verbatim}


    \begin{Verbatim}[commandchars=\\\{\}]
[1] "Estos son los productos que presentan un comportamiento de ruido blanco (H aproximadamente 0.5)"
[1] "Total de productos:"

    \end{Verbatim}

    2

    
    \begin{Verbatim}[commandchars=\\\{\}]
[1] "Nombre de los productos:"

    \end{Verbatim}

    \begin{enumerate*}
\item 'Arracacha'
\item 'Yuca'
\end{enumerate*}


    
    \hypertarget{construcciuxf3n-general-del-modelo-arima}{%
\section{ 3. Construcción general del modelo
Arima}\label{construcciuxf3n-general-del-modelo-arima}}

    \hypertarget{construcciuxf3n-del-modelo-garch}{%
\section{ 4. Construcción del modelo
Garch}\label{construcciuxf3n-del-modelo-garch}}

    \hypertarget{ecuaciones-de-pronuxf3stico}{%
\section{ 5. Ecuaciones de
pronóstico}\label{ecuaciones-de-pronuxf3stico}}

    \hypertarget{discusiuxf3n-y-observaciones-generales}{%
\section{ 6. Discusión y observaciones
generales}\label{discusiuxf3n-y-observaciones-generales}}

    \begin{Verbatim}[commandchars=\\\{\}]
{\color{incolor}In [{\color{incolor}2}]:} 
\end{Verbatim}


    \begin{Verbatim}[commandchars=\\\{\}]
also installing the dependency 'foreign'


    \end{Verbatim}

    \begin{Verbatim}[commandchars=\\\{\}]
package 'foreign' successfully unpacked and MD5 sums checked
package 'Amelia' successfully unpacked and MD5 sums checked

The downloaded binary packages are in
	C:\textbackslash{}Users\textbackslash{}Usuario\textbackslash{}AppData\textbackslash{}Local\textbackslash{}Temp\textbackslash{}RtmpemIx87\textbackslash{}downloaded\_packages

    \end{Verbatim}

    \hypertarget{creo-una-funciuxf3n-para-estimar-el-coeficiente-de-hurst}{%
\section{Creo una función para estimar el coeficiente de
Hurst}\label{creo-una-funciuxf3n-para-estimar-el-coeficiente-de-hurst}}

    \begin{Verbatim}[commandchars=\\\{\}]
{\color{incolor}In [{\color{incolor}83}]:} hurst\PYZus{}leo \PY{o}{\PYZlt{}\PYZhy{}} \PY{k+kr}{function}\PY{p}{(}x\PY{p}{)}\PY{p}{\PYZob{}}
           
           N \PY{o}{\PYZlt{}\PYZhy{}} \PY{k+kp}{length}\PY{p}{(}x\PY{p}{)}
           s \PY{o}{\PYZlt{}\PYZhy{}} x\PY{p}{[}\PY{l+m}{2}\PY{o}{:}N\PY{p}{]}
           t \PY{o}{\PYZlt{}\PYZhy{}}\PY{l+m}{1}\PY{o}{:}N
           par1\PY{o}{\PYZlt{}\PYZhy{}}\PY{l+m}{3}
           RS\PY{o}{\PYZlt{}\PYZhy{}}\PY{l+m}{0}\PY{o}{*}\PY{p}{(}\PY{l+m}{1}\PY{o}{:}par1\PY{p}{)}
           tau\PY{o}{\PYZlt{}\PYZhy{}}RS
           \PY{k+kr}{for}\PY{p}{(}i \PY{k+kr}{in} \PY{l+m}{1}\PY{o}{:}par1\PY{p}{)}\PY{p}{\PYZob{}}
             m \PY{o}{\PYZlt{}\PYZhy{}} \PY{k+kp}{floor}\PY{p}{(}N\PY{o}{/}\PY{p}{(}i\PY{p}{)}\PY{p}{)}
             R\PY{o}{\PYZlt{}\PYZhy{}}\PY{l+m}{0}\PY{o}{*}\PY{p}{(}\PY{l+m}{1}\PY{o}{:}i\PY{p}{)}
             S\PY{o}{\PYZlt{}\PYZhy{}}\PY{l+m}{0}\PY{o}{*}\PY{p}{(}\PY{l+m}{1}\PY{o}{:}i\PY{p}{)}
             Var\PY{o}{\PYZlt{}\PYZhy{}}R
             \PY{k+kr}{for}\PY{p}{(}j \PY{k+kr}{in} \PY{l+m}{1}\PY{o}{:}i\PY{p}{)}\PY{p}{\PYZob{}}
               smin\PY{o}{\PYZlt{}\PYZhy{}}\PY{l+m}{1}\PY{o}{+}\PY{p}{(}j\PY{l+m}{\PYZhy{}1}\PY{p}{)}\PY{o}{*}m
               smax\PY{o}{\PYZlt{}\PYZhy{}}j\PY{o}{*}m\PY{l+m}{\PYZhy{}1}
               r \PY{o}{\PYZlt{}\PYZhy{}} s\PY{p}{[}smin\PY{o}{:}smax\PY{p}{]}
               M\PY{o}{\PYZlt{}\PYZhy{}}\PY{k+kp}{mean}\PY{p}{(}r\PY{p}{)}
               x2\PY{o}{\PYZlt{}\PYZhy{}}r\PY{o}{\PYZhy{}}M
               V \PY{o}{\PYZlt{}\PYZhy{}} \PY{k+kp}{cumsum}\PY{p}{(}x2\PY{p}{)}
               R\PY{p}{[}j\PY{p}{]}\PY{o}{\PYZlt{}\PYZhy{}}\PY{k+kp}{max}\PY{p}{(}V\PY{p}{)}\PY{o}{\PYZhy{}}\PY{k+kp}{min}\PY{p}{(}V\PY{p}{)}
               Var\PY{p}{[}j\PY{p}{]}\PY{o}{\PYZlt{}\PYZhy{}}var\PY{p}{(}r\PY{p}{)}
               wj\PY{o}{\PYZlt{}\PYZhy{}}\PY{p}{(}\PY{l+m}{1}\PY{o}{\PYZhy{}}\PY{p}{(}j\PY{o}{/}\PY{p}{(}i\PY{l+m}{+1}\PY{p}{)}\PY{p}{)}\PY{p}{)}
               x22\PY{o}{\PYZlt{}\PYZhy{}}\PY{l+m}{0}
               \PY{k+kr}{for}\PY{p}{(}z \PY{k+kr}{in} \PY{l+m}{1}\PY{o}{:}\PY{p}{(}\PY{k+kp}{length}\PY{p}{(}x2\PY{p}{)}\PY{l+m}{\PYZhy{}1}\PY{p}{)}\PY{p}{)}\PY{p}{\PYZob{}}
                 x22\PY{p}{[}z\PY{p}{]}\PY{o}{\PYZlt{}\PYZhy{}}x2\PY{p}{[}z\PY{p}{]}\PY{o}{*}x2\PY{p}{[}z\PY{l+m}{+1}\PY{p}{]}
               \PY{p}{\PYZcb{}}
               cova\PY{o}{\PYZlt{}\PYZhy{}}\PY{k+kp}{sum}\PY{p}{(}x22\PY{p}{)}
               S\PY{p}{[}j\PY{p}{]} \PY{o}{\PYZlt{}\PYZhy{}} \PY{p}{(}Var\PY{p}{[}j\PY{p}{]} \PY{o}{+}  \PY{l+m}{2}\PY{o}{*}wj\PY{o}{*}cova\PY{o}{/}m\PY{p}{)}\PY{o}{\PYZca{}}\PY{p}{(}\PY{l+m}{0.5}\PY{p}{)}
             \PY{p}{\PYZcb{}}
             tau\PY{p}{[}i\PY{p}{]}\PY{o}{\PYZlt{}\PYZhy{}}m
             RS\PY{p}{[}i\PY{p}{]}\PY{o}{\PYZlt{}\PYZhy{}}\PY{k+kp}{mean}\PY{p}{(}R\PY{o}{/}S\PY{p}{)} 
           \PY{p}{\PYZcb{}}
           XX\PY{o}{\PYZlt{}\PYZhy{}} \PY{k+kt}{data.frame}\PY{p}{(}\PY{k+kp}{log10}\PY{p}{(}tau\PY{p}{)}\PY{p}{,}\PY{k+kp}{log10}\PY{p}{(}RS\PY{p}{)}\PY{p}{)}
           x \PY{o}{\PYZlt{}\PYZhy{}} XX\PY{o}{\PYZdl{}}log10.tau.
           y \PY{o}{\PYZlt{}\PYZhy{}} XX\PY{o}{\PYZdl{}}log10.RS.
           n \PY{o}{\PYZlt{}\PYZhy{}} \PY{k+kp}{nrow}\PY{p}{(}XX\PY{p}{)}
           xy \PY{o}{\PYZlt{}\PYZhy{}} x\PY{o}{*}y
           m2 \PY{o}{\PYZlt{}\PYZhy{}} \PY{p}{(}n\PY{o}{*}\PY{k+kp}{sum}\PY{p}{(}xy\PY{p}{)}\PY{o}{\PYZhy{}}\PY{k+kp}{sum}\PY{p}{(}x\PY{p}{)}\PY{o}{*}\PY{k+kp}{sum}\PY{p}{(}y\PY{p}{)}\PY{p}{)} \PY{o}{/} \PY{p}{(}n\PY{o}{*}\PY{k+kp}{sum}\PY{p}{(}x\PY{o}{\PYZca{}}\PY{l+m}{2}\PY{p}{)}\PY{o}{\PYZhy{}}\PY{k+kp}{sum}\PY{p}{(}x\PY{p}{)}\PY{o}{\PYZca{}}\PY{l+m}{2}\PY{p}{)}
           m2
           
           \PY{k+kr}{return}\PY{p}{(}m2\PY{p}{)}
         \PY{p}{\PYZcb{}}
\end{Verbatim}


    Son cargados los datos

    \begin{Verbatim}[commandchars=\\\{\}]
{\color{incolor}In [{\color{incolor}129}]:} DATA \PY{o}{\PYZlt{}\PYZhy{}} read.csv\PY{p}{(}file\PY{o}{=}\PY{l+s}{\PYZdq{}}\PY{l+s}{DATA\PYZus{}BUCARA.csv\PYZdq{}}\PY{p}{,} header\PY{o}{=}\PY{k+kc}{TRUE}\PY{p}{,} sep\PY{o}{=}\PY{l+s}{\PYZdq{}}\PY{l+s}{;\PYZdq{}}\PY{p}{)}\PY{c+c1}{\PYZsh{}carga desd archivo .csv ubicado en root}
          DATA \PY{o}{\PYZlt{}\PYZhy{}} \PY{k+kp}{as.data.frame}\PY{p}{(}DATA\PY{p}{)} \PY{c+c1}{\PYZsh{}transformación de los datos a formato Data Frame}
\end{Verbatim}


    Se suavizan logaritmicamente los datos

    \begin{Verbatim}[commandchars=\\\{\}]
{\color{incolor}In [{\color{incolor}105}]:} DATA\PY{p}{[}\PY{p}{,}\PY{l+m}{2}\PY{o}{:}\PY{k+kp}{length}\PY{p}{(}DATA\PY{p}{[}\PY{l+m}{1}\PY{p}{,}\PY{p}{]}\PY{p}{)}\PY{p}{]}\PY{o}{\PYZlt{}\PYZhy{}}\PY{k+kp}{log}\PY{p}{(}DATA\PY{p}{[}\PY{p}{,}\PY{l+m}{2}\PY{o}{:}\PY{k+kp}{length}\PY{p}{(}DATA\PY{p}{[}\PY{l+m}{1}\PY{p}{,}\PY{p}{]}\PY{p}{)}\PY{p}{]}\PY{p}{)}
\end{Verbatim}


    \begin{Verbatim}[commandchars=\\\{\}]
{\color{incolor}In [{\color{incolor}138}]:} \PY{c+c1}{\PYZsh{}\PYZsq{}ejemplo del modelo de pronóstico}
          BALts\PY{o}{\PYZlt{}\PYZhy{}}\PY{p}{(}ts\PY{p}{(}DATA\PY{p}{[}\PY{p}{,}\PY{l+m}{17}\PY{p}{]}\PY{p}{)}\PY{p}{)}
          plot\PY{p}{(}BALts\PY{p}{)}
\end{Verbatim}


    \begin{center}
    \adjustimage{max size={0.9\linewidth}{0.9\paperheight}}{output_47_0.png}
    \end{center}
    { \hspace*{\fill} \\}
    
    \begin{Verbatim}[commandchars=\\\{\}]
{\color{incolor}In [{\color{incolor}142}]:} ndiffs\PY{p}{(}BALts\PY{p}{,} test\PY{o}{=} \PY{k+kt}{c}\PY{p}{(}\PY{l+s}{\PYZdq{}}\PY{l+s}{kpss\PYZdq{}}\PY{p}{)}\PY{p}{)}
          ndiffs\PY{p}{(}BALts\PY{p}{,} test\PY{o}{=} \PY{k+kt}{c}\PY{p}{(}\PY{l+s}{\PYZdq{}}\PY{l+s}{adf\PYZdq{}}\PY{p}{)}\PY{p}{)}
          ndiffs\PY{p}{(}BALts\PY{p}{,} test\PY{o}{=} \PY{k+kt}{c}\PY{p}{(}\PY{l+s}{\PYZdq{}}\PY{l+s}{pp\PYZdq{}}\PY{p}{)}\PY{p}{)}
\end{Verbatim}


    1

    
    1

    
    0

    
    \begin{Verbatim}[commandchars=\\\{\}]
{\color{incolor}In [{\color{incolor}143}]:} auto.arima\PY{p}{(}BALts\PY{p}{)}
\end{Verbatim}


    
    \begin{verbatim}
Series: BALts 
ARIMA(1,1,2) with drift         

Coefficients:
         ar1      ma1      ma2   drift
      0.6870  -0.6951  -0.1439  2.9832
s.e.  0.1467   0.1538   0.0720  1.5807

sigma^2 estimated as 2374:  log likelihood=-1377.42
AIC=2764.84   AICc=2765.08   BIC=2782.65
    \end{verbatim}

    
    \begin{Verbatim}[commandchars=\\\{\}]
{\color{incolor}In [{\color{incolor}147}]:} Arl1\PY{o}{\PYZlt{}\PYZhy{}}arima\PY{p}{(}BALts\PY{p}{,} order \PY{o}{=} \PY{k+kt}{c}\PY{p}{(}\PY{l+m}{1}\PY{p}{,}\PY{l+m}{1}\PY{p}{,}\PY{l+m}{2}\PY{p}{)}\PY{p}{,} include.mean \PY{o}{=} \PY{k+kc}{TRUE}\PY{p}{)}
          Arl1
          f2\PY{o}{\PYZlt{}\PYZhy{}} forecast\PY{p}{(}Arl1\PY{p}{,} \PY{l+m}{5}\PY{p}{)}
          f2
          plot\PY{p}{(}f2\PY{p}{)}
          acf\PY{p}{(}residuals\PY{p}{(}f2\PY{p}{)}\PY{p}{)}  \PY{c+c1}{\PYZsh{}pg 782}
          pacf\PY{p}{(}residuals\PY{p}{(}f2\PY{p}{)}\PY{p}{)}
\end{Verbatim}


    
    \begin{verbatim}

Call:
arima(x = BALts, order = c(1, 1, 2), include.mean = TRUE)

Coefficients:
         ar1      ma1      ma2
      0.6369  -0.6333  -0.1424
s.e.  0.1804   0.1853   0.0692

sigma^2 estimated as 2367:  log likelihood = -1378.98,  aic = 2765.96
    \end{verbatim}

    
    
    \begin{verbatim}
    Point Forecast    Lo 80    Hi 80    Lo 95    Hi 95
262       1625.856 1563.511 1688.200 1530.508 1721.203
263       1624.289 1535.966 1712.613 1489.210 1759.368
264       1623.292 1519.859 1726.724 1465.105 1781.478
265       1622.656 1508.518 1736.794 1448.098 1797.215
266       1622.252 1499.667 1744.837 1434.774 1809.729
    \end{verbatim}

    
    \begin{center}
    \adjustimage{max size={0.9\linewidth}{0.9\paperheight}}{output_50_2.png}
    \end{center}
    { \hspace*{\fill} \\}
    
    \begin{center}
    \adjustimage{max size={0.9\linewidth}{0.9\paperheight}}{output_50_3.png}
    \end{center}
    { \hspace*{\fill} \\}
    
    \begin{center}
    \adjustimage{max size={0.9\linewidth}{0.9\paperheight}}{output_50_4.png}
    \end{center}
    { \hspace*{\fill} \\}
    
    \begin{Verbatim}[commandchars=\\\{\}]
{\color{incolor}In [{\color{incolor}134}]:} \PY{k+kp}{names}\PY{p}{(}X\PY{p}{)}
\end{Verbatim}


    \begin{enumerate*}
\item 'Fecha'
\item 'Ahuyama'
\item 'Ajo'
\item 'Arverja'
\item 'Cebolla.Cabezona'
\item 'Cebolla.junca'
\item 'Cilantro'
\item 'Cebada'
\item 'Maiz'
\item 'Frijol'
\item 'Habichuela'
\item 'Lechuga'
\item 'Perejil'
\item 'Pimenton'
\item 'Repollo.blanco'
\item 'Tomate'
\item 'Banano'
\item 'Lulo'
\item 'Mora'
\item 'Papaya'
\item 'Patilla'
\item 'Pinna'
\item 'Arracacha'
\item 'Papa'
\item 'Plátano'
\item 'Yuca'
\end{enumerate*}


    

    % Add a bibliography block to the postdoc
    
    
    
    \end{document}
